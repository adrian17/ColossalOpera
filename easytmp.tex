\section{Title}\label{title}

Approximate PI

\subsection{Difficulty}\label{difficulty}

Easy

\subsection{Tags}\label{tags}

math, Pi (constant)

\subsection{Description}\label{description}

The mathematical constant pi - the ratio of a circle's circumference to
its radius - is an irrational number. Approximations have been made -
first by hand and now by computers - for over 4000 years. The current
record is to 12.1 trillion digits!

Approximations start to fail after various decimal places. For instance,
the simple ``25/8'' approximation is good to only 2 decimal places,
while ``62832/20000'' is correct to 4 decimal places. Various algorithms
have been developed over the years that provide increasing accuracy.

For this challenge your task is to implement one or more of those
algorithms and approximate pi correctly to a specific number of digits.
You may \emph{NOT} for this challenge use your programming language or
system's built in definitions of pi (e.g.~System.Math.PI from .Net, or
M\_PI from math.h), you should not solve it using geometric functions
like \texttt{sin()} or \texttt{tan()} or the like, or ``find it'' by
downloading it from somewhere - that's straight up cheating.

\subsection{Challenge Input}\label{challenge-input}

You'll be given \emph{N}, a number of digits to which to correctly
approximate pi.

\begin{verbatim}
    4  (should be 3.1415 or 3.1416)
    6  (should be 3.141592 or 3.141593) 
    10 (should be 3.1415926535)
\end{verbatim}

\subsection{Scala Solution}\label{scala-solution}

\begin{verbatim}
def factorial(n:Int):Int = if (n==0) 1 else n * factorial(n-1)

def Ramanujan_Series(k:Int, sofar:Double): Double = {
    k match {
        case -1 => 1/(12*sofar)
        case _  => Ramanujan_Series(k - 1, sofar + ((math.pow(-1.0, k.toDouble) * factorial(6 * k) * (13591409 + 545140134 * k))/
         (factorial(3 * k) * math.pow(factorial(k), 3.0) * math.pow(640320, (1.5 + 3 * k)))))
     }
}  
\end{verbatim}

\section{Title}\label{title-1}

Balancing Words

\subsection{Difficulty}\label{difficulty-1}

Easy

\subsection{Tags}\label{tags-1}

letter sums, word games

\subsection{Description}\label{description-1}

Today we're going to balance words on one of the letters in them. We'll
use the position and letter itself to calculate the weight around the
balance point. A word can be balanced if the weight on either side of
the balance point is equal. Not all words can be balanced, but those
that can are interesting for this challenge.

The formula to calculate the weight of the word is to look at the letter
position in the English alphabet (so A=1, B=2, C=3 \ldots{} Z=26) as the
letter weight, then multiply that by the distance from the balance
point, so the first letter away is multiplied by 1, the second away by
2, etc.

As an example:

STEAD balances at T: 1 * S(19) = 1 * E(5) + 2 * A(1) + 3 * D(4))

More info
\href{http://www.questrel.com/records.html\#spelling_alphabetical_order_entire_word_balance_points}{here}
on the Questrel site.

\subsection{Input Description}\label{input-description}

You'll be given a series of English words. Example:

\begin{verbatim}
STEAD
\end{verbatim}

\subsection{Output Description}\label{output-description}

Your program or function should emit the words split by their balance
point and the weight on either side of the balance point. Example:

\begin{verbatim}
S T EAD - 19
\end{verbatim}

This indicates that the T is the balance point and that the weight on
either side is 19.

\subsection{Challenge Input}\label{challenge-input-1}

\begin{verbatim}
CONSUBSTANTIATION
WRONGHEADED
UNINTELLIGIBILITY
\end{verbatim}

\subsection{Challenge Output}\label{challenge-output}

\begin{verbatim}
CONSUBST A NTIATION - 456
WRO N GHEADED - 120
UNINTELL I GIBILITY - 521
\end{verbatim}

\subsection{Notes}\label{notes}

This was found on a word games page suggested by /u/cDull, thanks! If
you have your own idea for a challenge, submit it to
/r/DailyProgrammer\_Ideas, and there's a good chance we'll post it.

\subsection{Scala Solution}\label{scala-solution-1}

\begin{verbatim}
def balance(word:String): (String, String, String, Int) = {
  def loop(word:String, n:Int):(Int, Int) = {
    n+word.length match {
      case 1 =>       (0, -1)
      case _ =>
        val p = word.map(_.toInt-64).zip(n to (word.length+n-1)).map(x=>x._1*x._2).partition(_>0)
        val lhs = p._1.sum
        val rhs = p._2.sum
        (lhs + rhs == 0) match {
          case true  => (lhs, (n to (word.length+n-1)).indexOf(0))
          case false => loop(word, n-1)
        }
    }
  }
  val b = loop(word.toUpperCase, 0)
  b._1 match {
    case 0 => ("", "", "", -1)
    case _ => (word.substring(0, b._2), word(b._2).toString, word.substring(b._2+1, word.length), b._1)
  }
}

// how many words can be balanced?
val bwords = scala.io.Source.
              fromFile("/usr/share/dict/words").
              getLines.
              map(balance(_)).
              filter(_._1 != "")
\end{verbatim}

\section{Title}\label{title-2}

Broken Keyboard

\subsection{Difficulty}\label{difficulty-2}

Easy

\subsection{Tags}\label{tags-2}

words, keyboard

\subsection{Description}\label{description-2}

Help! My keyboard is broken, only a few keys work any more. If I tell
you what keys work, can you tell me what words I can write?

(You should use the trusty enable1.txt file, or
\texttt{/usr/share/dict/words} to chose your valid English words from.)

\subsection{Input Description}\label{input-description-1}

You'll be given a line with a single integer on it, telling you how many
lines to read. Then you'll be given that many lines, each line a list of
letters representing the keys that work on my keyboard. Example:

\begin{verbatim}
3
abcd
qwer
hjklo
\end{verbatim}

\subsection{Output Description}\label{output-description-1}

Your program should emit the longest valid English language word you can
make for each keyboard configuration.

\begin{verbatim}
abcd = bacaba
qwer = ewerer
hjklo = kolokolo
\end{verbatim}

\subsection{Challenge Input}\label{challenge-input-2}

\begin{verbatim}
4
edcf
bnik
poil
vybu
\end{verbatim}

\subsection{Challenge Output}\label{challenge-output-1}

\begin{verbatim}
edcf = deedeed
bnik = bikini
poil = pililloo
vybu = bubby
\end{verbatim}

\subsection{Scala Solution}\label{scala-solution-2}

Uses regexes

\begin{verbatim}
val words = io.Source.fromFile("/usr/share/dict/words").mkString.split("\n").toList
def typewriter(keys:String): String = words.filter(("[" + keys + "]+").r.replaceAllIn(_,"")=="").sortBy(x=>x.length).last
\end{verbatim}

\section{Title}\label{title-3}

Cellular Automata: Rule 90

\subsection{Difficulty}\label{difficulty-3}

Easy

\subsection{Tags}\label{tags-3}

cellular automata, dynamical systems, automata

\subsection{Description}\label{description-3}

The development of (cellular
automata){[}https://en.wikipedia.org/wiki/Cellular\_automaton{]} (CA)
systems is typically attributed to Stanisław Ulam and John von Neumann,
who were both researchers at the Los Alamos National Laboratory in New
Mexico in the 1940s. Ulam was studying the growth of crystals and von
Neumann was imagining a world of self-replicating robots. That's right,
robots that build copies of themselves. Once we see some examples of CA
visualized, it'll be clear how one might imagine modeling crystal
growth; the robots idea is perhaps less obvious. Consider the design of
a robot as a pattern on a grid of cells (think of filling in some
squares on a piece of graph paper). Now consider a set of simple rules
that would allow that pattern to create copies of itself on that grid.
This is essentially the process of a CA that exhibits behavior similar
to biological reproduction and evolution. (Incidentally, von Neumann's
cells had twenty-nine possible states.) Von Neumann's work in
self-replication and CA is conceptually similar to what is probably the
most famous cellular automaton: Conways ``Game of Life,'' sometimes seen
as a screen saver. CA has been pushed very hard by Stephen Wolfram
(e.g.~Mathematica, Worlram Alpha, and ``A New Kind of Science'').

CA has a number of simple ``rules'' that define system behavior, like
``If my neighbors are both active, I am inactive'' and the like. The
rules are all given numbers, but they're not sequential for historical
reasons.

The subject rule for this challenge, Rule 90, is one of the simplest, a
simple neighbor XOR. That is, in a 1 dimensional CA system (e.g.~a
line), the next state for the cell in the middle of 3 is simply the
result of the XOR of its left and right neighbors. E.g. ``000'' becomes
``1'' in the next state, ``100'' becomes ``1'' in the next state and so
on. You traverse the given line in windows of 3 cells and calculate the
rule for the next iteration of the following row's center cell based on
the current one while the two outer cells are influenced by their
respective neighbors. Here are the rules showing the conversion from one
set of cells to another:

\begin{longtable}[c]{@{}llllllll@{}}
\toprule\addlinespace
``111'' & ``101'' & ``010'' & ``000'' & ``110'' & ``100'' & ``011'' &
``001''
\\\addlinespace
\midrule\endhead
0 & 0 & 0 & 0 & 1 & 1 & 1 & 1
\\\addlinespace
\bottomrule
\end{longtable}

\subsection{Input Description}\label{input-description-2}

You'll be given an input line as a series of 0s and 1s. Example:

\begin{verbatim}
1101010
\end{verbatim}

\subsection{Output Description}\label{output-description-2}

Your program should emit the states of the celular automata for 25
steps. Example from above, in this case I replaced 0 with a blank and a
1 with an X:

\begin{verbatim}
xx x x
xx    x
xxx  x
x xxx x
  x x
 x   x
\end{verbatim}

\subsection{Challenge Input}\label{challenge-input-3}

\begin{verbatim}
00000000000000000000000000000000000000000000000001000000000000000000000000000000000000000000000000
\end{verbatim}

\subsection{Challenge Output}\label{challenge-output-2}

I chose this input because it's one of the most well known, it yields a
Serpinski triangle, a well known fractcal.

\begin{verbatim}
                                             x
                                            x x
                                           x   x
                                          x x x x
                                         x       x
                                        x x     x x
                                       x   x   x   x
                                      x x x x x x x x
                                     x               x
                                    x x             x x
                                   x   x           x   x
                                  x x x x         x x x x
                                 x       x       x       x
                                x x     x x     x x     x x
                               x   x   x   x   x   x   x   x
                              x x x x x x x x x x x x x x x x
                             x                               x
                            x x                             x x
                           x   x                           x   x
                          x x x x                         x x x x
                         x       x                       x       x
                        x x     x x                     x x     x x
                       x   x   x   x                   x   x   x   x
                      x x x x x x x x                 x x x x x x x x
                     x               x               x               x
                    x x             x x             x x             x x
\end{verbatim}

\subsection{Scala Solution}\label{scala-solution-3}

\begin{verbatim}
def rule90(row:String): String = {
    def loop(s:String): String = {
        s match {
            case "111" | "101" | "010" | "000" => "0"
            case "110" | "100" | "011" | "001" => "1"
        }
    }
    ("0" + row + "0").sliding(3).map(loop(_)).toList.mkString
}

def solution(s:String, n:Int) = {
    var row = s
    for (_ <- (0 to n)) {
        println(row.replace("0", " ").replace("1", "x"))
        row = rule90(row)
    }
}
\end{verbatim}

\section{Title}\label{title-4}

Letters in Alphabetical Order

\subsection{Difficulty}\label{difficulty-4}

Easy

\subsection{Tags}\label{tags-4}

word games, alphabet

\subsection{Description}\label{description-4}

A handful of words have their letters in alphabetical order, that is
nowhere in the word do you change direction in the word if you were to
scan along the English alphabet. An example is the word ``almost'',
which has its letters in alphabetical order.

Your challenge today is to write a program that can determine if the
letters in a word are in alphabetical order.

As a bonus, see if you can find words spelled in \emph{reverse}
alphebatical order.

\subsection{Input Description}\label{input-description-3}

You'll be given one word per line, all in standard English. Examples:

\begin{verbatim}
almost
cereal
\end{verbatim}

\subsection{Output Description}\label{output-description-3}

Your program should emit the word and if it is in order or not.
Examples:

\begin{verbatim}
almost IN ORDER
cereal NOT IN ORDER
\end{verbatim}

\subsection{Challenge Input}\label{challenge-input-4}

\begin{verbatim}
billowy
biopsy
chinos
defaced
chintz
sponged
bijoux
abhors
fiddle
begins
chimps
wronged
\end{verbatim}

\subsection{Challenge Output}\label{challenge-output-3}

\begin{verbatim}
billowy IN ORDER
biopsy IN ORDER
chinos IN ORDER
defaced NOT IN ORDER
chintz IN ORDER
sponged REVERSE ORDER 
bijoux IN ORDER
abhors IN ORDER
fiddle NOT IN ORDER
begins IN ORDER
chimps IN ORDER
wronged REVERSE ORDER
\end{verbatim}

\subsection{Scala Solution}\label{scala-solution-4}

\begin{verbatim}
def alphabetical(word:String): Boolean = word.map(_.toInt).sorted == word.map(_.toInt)
def rev_alphabetical(word:String): Boolean = word.map(_.toInt).sorted.reverse == word.map(_.toInt)

def main(word:String) = {
    if (alphabetical(word) == true) 
        println(word + " IN ORDER")
    else if (rev_alphabetical(word) == true) 
        println(word + " IN REVERSE ORDER")        
    else
        println(word + " NOT IN ORDER")
}
\end{verbatim}

\section{Title}\label{title-5}

Making numbers palindromic

\subsection{Difficulty}\label{difficulty-5}

Easy

\subsection{Tags}\label{tags-5}

palindrome, number

\subsection{Description}\label{description-5}

To covert nearly any number into a palindromic number you operate by
reversing the digits and adding and then repeating the steps until you
get a palindromic number. Some require many steps.

e.g.~24 gets palindromic after 1 steps: 66 -\textgreater{} 24 + 42 = 66

while 28 gets palindromic after 2 steps: 121 -\textgreater{} 28 + 82 =
110, so 110 + 11 (110 reversed) = 121.

Note that, as an example, 196 never gets palindromic (at least according
to researchers, at least never in reasonable time). Several numbers
never appear to approach being palindromic.

\subsection{Input Description}\label{input-description-4}

You will be given a number, one per line. Example:

\begin{verbatim}
11
68
\end{verbatim}

\subsection{Output Description}\label{output-description-4}

You will describe how many steps it took to get it to be palindromic,
and what the resulting palindrome is. Example:

\begin{verbatim}
11 gets palindromic after 0 steps: 11
68 gets palindromic after 3 steps: 1111
\end{verbatim}

\subsection{Challenge Input}\label{challenge-input-5}

\begin{verbatim}
123
286
196196871
\end{verbatim}

\subsection{Challenge Input Solution}\label{challenge-input-solution}

\begin{verbatim}
123 gets palindromic after 1 steps: 444
286 gets palindromic after 23 steps: 8813200023188
196196871 gets palindromic after 45 steps: 4478555400006996000045558744
\end{verbatim}

\subsection{Note}\label{note}

Bonus: see which input numbers, through 1000, yield identical
palindromes.

Bonus 2: See which numbers don't get palindromic in under 10000 steps.
Numbers that never converge are called Lychrel numbers.

\subsection{Scala Solution}\label{scala-solution-5}

\begin{verbatim}
def reverse(n:Long): Long = n.toString.reverse.toLong

def palindrome(n:Long): Boolean =  n == reverse(n)

def loop(n:Long, steps:Int): (Long, Int) = {
    palindrome(n) match {
        case true  => (n, steps)
        case false => loop(reverse(n)+n, steps + 1)
    }
}
\end{verbatim}

\section{Title}\label{title-6}

Pronouncing Hexidecimal

\subsection{Difficulty}\label{difficulty-6}

Easy

\subsection{Tags}\label{tags-6}

hexidecimal, Silicon Valley

\subsection{Description}\label{description-6}

The HBO network show ``Silicon Valley'' has introduced a way to
pronounce hex.

\begin{verbatim}
Kid: Here it is: Bit... soup. It's like alphabet soup, BUT... it's ones and zeros instead of letters.
Bachman: {silence}
Kid: 'Cause it's binary? You know, binary's just ones and zeroes.
Bachman: Yeah, I know what binary is. Jesus Christ, I memorized the hexadecimal times tables when I was fourteen writing machine code. Okay? Ask me what nine times F is. It's fleventy-five. I don't need you to tell me what binary is.
\end{verbatim}

Not ``eff five'', fleventy. \texttt{0xF0} is now fleventy. Awesome.
Above a full byte you add ``bitey'' to the name. The hexidecimal
pronunciation rules:

\begin{longtable}[c]{@{}ll@{}}
\toprule\addlinespace
HEX PLACE VALUE & WORD
\\\addlinespace
\midrule\endhead
0xA0 & ``Atta''
\\\addlinespace
0xB0 & ``Bibbity''
\\\addlinespace
0xC0 & ``City''
\\\addlinespace
0xD0 & ``Dickety''
\\\addlinespace
0xE0 & ``Ebbity''
\\\addlinespace
0xF0 & ``Fleventy''
\\\addlinespace
0xA000 & ``Atta-bitey''
\\\addlinespace
0xB000 & ``Bibbity-bitey''
\\\addlinespace
0xC000 & ``City-bitey''
\\\addlinespace
0xD000 & ``Dickety-bitey''
\\\addlinespace
0xE000 & ``Ebbity-bitey''
\\\addlinespace
0xF000 & ``Fleventy-bitey''
\\\addlinespace
\bottomrule
\end{longtable}

Combinations like \texttt{0xABCD} are then spelled out ``atta-bee bitey
city-dee''.

For this challenge you'll be given some hex strings and asked to
pronounce them.

\subsection{Input Description}\label{input-description-5}

You'll be given a list of hex values, one per line. Examples:

\begin{verbatim}
0xF5
0xB3
0xE4
0xBBBB
0xA0C9 
\end{verbatim}

\subsection{Output Description}\label{output-description-5}

Your program should emit the pronounced hex. Examples from above:

\begin{verbatim}
0xF5 "fleventy-five"
0xB3 "bibbity-three"
0xE4 "ebbity-four"
0xBBBB "bibbity-bee bitey bibbity-bee"
0xA0C9 "atta-bitey city-nine"
\end{verbatim}

\section{Title}\label{title-7}

Ruth-Aaron Pairs

\subsection{Difficulty}\label{difficulty-7}

Easy

\subsection{Tags}\label{tags-7}

prime numbers, prime factors, number theory

\subsection{Description}\label{description-7}

In mathematics, a Ruth--Aaron pair consists of two consecutive integers
(e.g.~714 and 715) for which the sums of the \emph{distinct} prime
factors of each integer are equal. For example, we know that (714, 715)
is a valid Ruth-Aaron pair because its distinct prime factors are:

\begin{verbatim}
714 = 2 * 3 * 7 * 17
715 = 5 * 11 * 13
\end{verbatim}

and the sum of those is both 29:

\begin{verbatim}
2 + 3 + 7 + 17 = 5 + 11 + 13 = 29
\end{verbatim}

The name was given by Carl Pomerance, a mathematician at the University
of Georgia, for Babe Ruth and Hank Aaron, as Ruth's career
regular-season home run total was 714, a record which Aaron eclipsed on
April 8, 1974, when he hit his 715th career home run. A student of one
of Pomerance's colleagues noticed that the sums of the prime factors of
714 and 715 were equal.

For a little more on it, see
\href{http://mathworld.wolfram.com/Ruth-AaronPair.html}{MathWorld}.

Your task today is to determine if a pair of numbers is indeed a valid
Ruth-Aaron pair.

\subsection{Input Description}\label{input-description-6}

You'll be given a single integer \emph{N} on one line to tell you how
many pairs to read, and then that many pairs as two-tuples. For example:

\begin{verbatim}
3
(714,715)
(77,78)
(20,21)
\end{verbatim}

\subsection{Output Description}\label{output-description-6}

Your program should emit if the numbers are valid Ruth-Aaron pairs.
Example:

\begin{verbatim}
(714,715) VALID
(77,78) VALID
(20,21) NOT VALID
\end{verbatim}

\subsection{Chalenge Input}\label{chalenge-input}

\begin{verbatim}
4
(5,6) 
(2107,2108) 
(492,493) 
(128,129) 
\end{verbatim}

\subsection{Challenge Output}\label{challenge-output-4}

\begin{verbatim}
(5,6) VALID
(2107,2108) VALID
(492,493) VALID
(128,129) NOT VALID
\end{verbatim}

\subsection{Scala Solution}\label{scala-solution-6}

\begin{verbatim}
def factorize(x: Int): List[Int] = {
  @tailrec
  def foo(x: Int, a: Int = 2, list: List[Int] = Nil): List[Int] = a*a > x match {
    case false if x % a == 0 => foo(x / a, a    , a :: list)
    case false               => foo(x    , a + 1, list)
    case true                => x :: list
  }
  foo(x)
}

def RA(a:Int, b:Int): Boolean = def RA(a:Int, b:Int): Boolean =  factorize(a).toSet.sum == factorize(b).toSet.sum
\end{verbatim}

\subsection{CSharp Solution}\label{csharp-solution}

\begin{verbatim}
using System;
using System.Collections.Generic;
using System.Linq;

class Solution {
    public static void Main(string[] args) {
        int ra;
        bool t = int.TryParse(args[0], out ra);
        List<int> aints = PrimeFactors(ra);
        List<int> bints = PrimeFactors(ra+1);
        Console.WriteLine("{0} and {1} => {2}", ra, ra+1, aints.Sum() == bints.Sum());
    }

    private static List<int> PrimeFactors(int n) {
        List<int> ints = new List<int>();
        for (int i = 2; i <= n; i++) {
            if (IsPrime(i) && (n%i == 0)) {
                ints.Add(i);
            }
        }
        return ints;
    }

    private static bool IsPrime(int n)
    {
        bool prime = true;
        int i = 2;
        do {
            prime &= (i == n || n%i != 0);
            i += (i == 2 ? 1 : 2);
        } while (i <= n);

        return prime;
    }
}
\end{verbatim}

\section{Title}\label{title-8}

Shuffling a List

\subsection{Difficulty}\label{difficulty-8}

Easy

\subsection{Tags}\label{tags-8}

algorithm, shuffle, Fisher-Yates, Faro

\subsection{Description}\label{description-8}

We've had our fair share of sorting algorithms, now let's do a
\emph{shuffling} challenge. In this challenge, your challenge is to take
a list of inputs and change around the order in random ways. Think about
shuffling cards - can your program shuffle cards?

\subsection{Input Description}\label{input-description-7}

You'll be given a list of values - integers, letters, words - in one
order. The input list will be space separated. Example:

\begin{verbatim}
1 2 3 4 5 6 7 8 
\end{verbatim}

\subsection{Output Description}\label{output-description-7}

Your program should emit the values in any non-sorted order; sequential
runs of the program or function should yield different outputs. You
should maximize the disorder if you can. From our example:

\begin{verbatim}
7 5 4 3 1 8 2 6
\end{verbatim}

\subsection{Challenge Input}\label{challenge-input-6}

\begin{verbatim}
apple blackberry cherry dragonfruit grapefruit kumquat mango nectarine persimmon raspberry raspberry
a e i o u
\end{verbatim}

\subsection{Challenge Output}\label{challenge-output-5}

Examples only, this is all about shuffling

\begin{verbatim}
raspberry blackberry nectarine kumquat grapefruit cherry raspberry apple mango persimmon dragonfruit
e a i o u
\end{verbatim}

\subsection{Bonus}\label{bonus}

Check out the Faro shuffle and the Fisher Yates shuffles, which are
algorithms for specific shuffles. Shuffling has some interesting
mathematical properties.

\subsection{Scala Solution}\label{scala-solution-7}

\begin{verbatim}
def fischer_yates_shuffle(l:List[Int]): List[Int] = {   
    def loop(l:List[Int], n:Int): List[Int] = {
        (l.length == n) match {
            case true   => l
            case false  => val i = (scala.math.random*l.length).toInt
                           l.slice(0, n) ++ List(l(i)) ++ l.slice(n+1,i) ++ List(l(n)) ++ l.slice(i+1,l.length)
        }
    }
    loop(l, 0)
}

def faro_shuffle(l:List[Int], steps:Int): List[Int] = {
    def loop(l:List[Int], n:Int): List[Int] = {
        (n == 0) match {
            case true  =>   l
            case false =>   val (a,b) = (l.slice(0, l.length/2), l.slice(l.length/2, l.length))
                            if (a.length != b.length) {
                                loop(a.zip(b).flatMap(x => List(x._1, x._2)) ++ List(b.last), n-1)
                            } else {
                                loop(a.zip(b).flatMap(x => List(x._1, x._2)), n-1)
                            }
        }
    }
    loop(l, steps)
}
\end{verbatim}

\section{Title}\label{title-9}

Playing the Stock Market

\subsection{Difficulty}\label{difficulty-9}

Easy

\subsection{Tags}\label{tags-9}

stock market

\subsection{Description}\label{description-9}

Let's assume I'm playing the stock market - buy low, sell high. I'm a
day trader, so I want to get in and out of a stock before the day is
done, and I want to time my trades so that I make the biggest gain
possible.

The market has a rule that won't let me buy and sell in a pair of ticks
- I have to wait for at least one tick to go buy. And obviously I can't
buy in the future and sell in the past.

So, given a list of stock price ticks for the day, can you tell me what
trades I should make to maximize my gain within the constraints of the
market? Remember - buy low, sell high, and you can't sell before you
buy.

\subsection{Input Description}\label{input-description-8}

You'll be given a list of stock prices as a 2 decimal float (dollars and
cents), listed in chronological order. Example:

\begin{verbatim}
19.35 19.30 18.88 18.93 18.95 19.03 19.00 18.97 18.97 18.98
\end{verbatim}

\subsection{Output Description}\label{output-description-8}

Your program should emit the two trades in chronological order - what
you think I should buy at and sell at. Example:

\begin{verbatim}
18.88 19.03
\end{verbatim}

\subsection{Challenge Input}\label{challenge-input-7}

\begin{verbatim}
9.20 8.03 10.02 8.08 8.14 8.10 8.31 8.28 8.35 8.34 8.39 8.45 8.38 8.38 8.32 8.36 8.28 8.28 8.38 8.48 8.49 8.54 8.73 8.72 8.76 8.74 8.87 8.82 8.81 8.82 8.85 8.85 8.86 8.63 8.70 8.68 8.72 8.77 8.69 8.65 8.70 8.98 8.98 8.87 8.71 9.17 9.34 9.28 8.98 9.02 9.16 9.15 9.07 9.14 9.13 9.10 9.16 9.06 9.10 9.15 9.11 8.72 8.86 8.83 8.70 8.69 8.73 8.73 8.67 8.70 8.69 8.81 8.82 8.83 8.91 8.80 8.97 8.86 8.81 8.87 8.82 8.78 8.82 8.77 8.54 8.32 8.33 8.32 8.51 8.53 8.52 8.41 8.55 8.31 8.38 8.34 8.34 8.19 8.17 8.16
\end{verbatim}

\subsection{Challenge Output}\label{challenge-output-6}

\begin{verbatim}
8.03 9.34
\end{verbatim}

\subsection{Python Quote Generator}\label{python-quote-generator}

\begin{verbatim}
import random

def stockprices(init, sofar=[]): 
    if len(sofar) == 100: return sofar
    else:
        sofar.append(init + (random.paretovariate(init)-random.paretovariate(init))*0.5)
        return stockprices(sofar[-1], sofar)

print ' '.join(map(lambda x: '%.2f' % x, stockprices(8, [])))
\end{verbatim}

\subsection{Scala Solution}\label{scala-solution-8}

\begin{verbatim}
def pick(quotes:String) = {
    def loop(quotes:List[Double], best:(Double, Double)): (Double, Double) = {
        quotes.length match {
            case 2 => best
            case _ => {
                val biggest = quotes.tail.tail.map(x => ((quotes.head, x), x-quotes.head)).maxBy(_._2)
                if (biggest._2 > (best._2-best._1)) {
                    loop(quotes.tail, biggest._1)
                } else {
                    loop(quotes.tail, best)
                }
            }
        }
    }
    loop(quotes.split(" ").map(_.toDouble).toList, (0.0, 0.0))
}
\end{verbatim}

\section{Title}\label{title-10}

Thue-Morse Sequence Generator

\subsection{Tags}\label{tags-10}

number theory, integer sequence, binary sequence, Thue-Morse, infinite
sequence

\subsection{Description}\label{description-10}

The Thue-Morse sequence is a binary sequence (of 0s and 1s) that never
repeats. It is obtained by starting with 0 and successively calculating
the Boolean complement of the sequence so far. It turns out that doing
this yields an infinite, non-repeating sequence. This procedure yields 0
then 01, 0110, 01101001, 0110100110010110, and so on.

See the
\href{http://en.wikipedia.org/wiki/Thue\%E2\%80\%93Morse_sequence}{Thue-Morse
Wikipedia Article} for more information.

\subsection{Input}\label{input}

Nothing.

\subsection{Output}\label{output}

Output the 0 to 6th order Thue-Morse Sequences.

\subsection{Example}\label{example}

\begin{verbatim}
nth         Sequence
===========================================================================
0           0
1           01
2           0110
3           01101001
4           0110100110010110
5           01101001100101101001011001101001
6           0110100110010110100101100110100110010110011010010110100110010110
\end{verbatim}

\subsection{Extra Challenge}\label{extra-challenge}

Be able to output any nth order sequence. Display the Thue-Morse
Sequences for 100.

Note: Due to the size of the sequence it seems people are crashing
beyond 25th order or the time it takes is very long. So how long until
you crash. Experiment with it.

\subsection{Fsharp Solution}\label{fsharp-solution}

\begin{verbatim}
let rec A3061 (L) =
    match (List.head L, List.tail L) with
    | (1, []) -> [ 0 ]
    | (0, []) -> [ 1 ]
    | (1, _ ) -> [ 0 ] @ A3061 (List.tail L)
    | (0, _ ) -> [ 1 ] @ A3061 (List.tail L)

let thuemorse (n:int) = 
    let mutable L = [0]
    for i in [1..n] do
        L <- L @ A3061 L
    L
\end{verbatim}

\section{Title}\label{title-11}

Vampire Numbers

\subsection{Difficulty}\label{difficulty-10}

Easy

\subsection{Tags}\label{tags-11}

composite numbers, number theory

\subsection{Description}\label{description-11}

A vampire number \emph{v} is a number \emph{v=xy} with an even number
\emph{n} of digits formed by multiplying a pair of \emph{n}/2-digit
numbers (where the digits are taken from the original number in any
order) \emph{x} and \emph{y} together. Pairs of trailing zeros are not
allowed. If \emph{v} is a vampire number, then \emph{x} and \emph{y} are
called its ``fangs.''

Additional information can be found
\href{http://www.primepuzzles.net/puzzles/puzz_199.htm}{here}.

\subsection{Input Description}\label{input-description-9}

Two digits on one line indicating \emph{n}, the number of digits in the
number to factor and find if it is a vampire number, and \emph{m}, the
number of fangs. Example:

\begin{verbatim}
4 2
\end{verbatim}

\subsection{Output Description}\label{output-description-9}

A list of all vampire numbers of \emph{n} digits, you should emit the
number and its factors (or ``fangs''). Example:

\begin{verbatim}
1260=21*60
1395=15*93
1435=41*35
1530=51*30
1827=87*21
2187=27*81
6880=86*80
\end{verbatim}

\subsection{Challenge Input}\label{challenge-input-8}

\begin{verbatim}
6 3
\end{verbatim}

\subsection{Challenge Input Solution}\label{challenge-input-solution-1}

\begin{verbatim}
114390=41*31*90
121695=21*61*95
127428=21*74*82
127680=21*76*80
127980=20*79*81
137640=31*74*60
139500=31*90*50
163680=66*31*80
178920=71*90*28
197925=91*75*29
198450=81*49*50
247680=40*72*86
294768=46*72*89
376680=73*60*86
397575=93*75*57
457968=56*94*87
479964=74*94*69
498960=99*84*60
\end{verbatim}

\subsection{Scala Solution}\label{scala-solution-9}

\begin{verbatim}
object VampireNumbers {
  def product(list: List[Int]): Int = list.foldLeft(1)(_*_)

  def vampire(n:Int, fangs:Int):List[(Int, List[Int])] ={
    n.
     toString.
     map(_.toString.toInt).
     permutations.
     map(_.grouped(2).map(_.mkString.toInt).toList).
     map(x=>(product(x),x)).
     filter(_._1==n).
     toList
  }

  def main(argc:Int, argv:Array[String]) = {
    val start = scala.math.pow(10, argv(1).toInt-1).toInt
    val end = scala.math.pow(10, argv(1).toInt).toInt-1
    val fangs = argv(2).toInt
    (start to end).map(x => vampire(x, fangs)).filter(_.length > 0).foreach(println)
  }
}
\end{verbatim}

\section{Title}\label{title-12}

Abundant and Deficient Numbers

\subsection{Difficulty}\label{difficulty-11}

Easy

\subsection{Tags}\label{tags-12}

number theory, divisor function

\subsection{Description}\label{description-12}

In number theory, a deficient or \textbf{deficient number} is a number n
for which the sum of divisors \emph{sigma(n)\textless{}2n}, or,
equivalently, the sum of proper divisors (or aliquot sum) \emph{s(n) n
def deficient(n:Int): Boolean = 2}divisors(n).sum \textless{} n def
main(n:Int): String = \{ if (abundant(n)) \{ return ``abundant by'' +
abundance(n) \} if (deficient(n)) \{ return ``deficient'' \} return
``neither'' \}

\section{Title}\label{title-13}

Detecting Alliteration

\subsection{Difficulty}\label{difficulty-12}

Easy

\subsection{Tags}\label{tags-13}

word games, alliteration, poetic device

\subsection{Description}\label{description-13}

Alliteration is defined as ``the occurrence of the same letter or sound
at the beginning of adjacent or closely connected words.'' It's a
stylistic literary device identified by the repeated sound of the first
consonant in a series of multiple words, or the repetition of the same
sounds or of the same kinds of sounds at the beginning of words or in
stressed syllables of a phrase. The first known use of the word to refer
to a literary device occurred around 1624. A simple example is ``Peter
Piper Picked a Peck of Pickled Peppers''.

\subsection{Note on Stop Words}\label{note-on-stop-words}

The following are some of the simplest English ``stop words'', words too
common and uninformative to be of much use. In the case of Alliteration,
they can come in between the words of interest (as in the Peter Piper
example):

\begin{verbatim}
I 
a 
about 
an 
are 
as 
at 
be 
by 
com 
for 
from
how
in 
is 
it 
of 
on 
or 
that
the 
this
to 
was 
what 
when
where
who 
will 
with
the
\end{verbatim}

\subsection{Sample Input}\label{sample-input}

You'll be given an integer on a line, telling you how many lines follow.
Then on the subsequent ines, you'll be given a sentence, one per line.
Example:

\begin{verbatim}
3
Peter Piper Picked a Peck of Pickled Peppers
Bugs Bunny likes to dance the slow and simple shuffle
You'll never put a better bit of butter on your knife
\end{verbatim}

\subsection{Sample Output}\label{sample-output}

Your program should emit the words from each sentence that form the
group of alliteration. Example:

\begin{verbatim}
Peter Piper Picked Peck Pickled Peppers
Bugs Bunny      slow simple shuffle
better bit butter
\end{verbatim}

\subsection{Challenge Input}\label{challenge-input-9}

\begin{verbatim}
8
The daily diary of the American dream
For the sky and the sea, and the sea and the sky
Three grey geese in a green field grazing, Grey were the geese and green was the grazing.
But a better butter makes a batter better.
"His soul swooned slowly as he heard the snow falling faintly through the universe and faintly falling, like the descent of their last end, upon all the living and the dead."
Whisper words of wisdom, let it be.
They paved paradise and put up a parking lot.
So what we gonna have, dessert or disaster?
\end{verbatim}

\subsection{Challenge Output}\label{challenge-output-7}

\begin{verbatim}
daily diary
sky sea
grey geese green grazing
better butter batter better
soul swooned slowly
whisper words wisdom
paved paradise
dessert disaster
\end{verbatim}

\section{Title}\label{title-14}

Anagram Detector

\subsection{Difficulty}\label{difficulty-13}

Easy

\subsection{Tags}\label{tags-14}

word games, anagram

\subsection{Description}\label{description-14}

An anagram is a form of word play, where you take a word (or set of
words) and form a different word (or different set of words) that use
the same letters, just rearranged. All words must be valid spelling, and
shuffling words around doesn't count.

Some serious word play aficionados find that some anagrams can contain
meaning, like ``Clint Eastwood'' and ``Old West Action'', or ``silent''
and ``listen''.

Someone once said, ``All the life's wisdom can be found in anagrams.
Anagrams never lie.'' How they don't lie is beyond me, but there you go.

Punctuation and capitalization don't matter.

\subsection{Input Description}\label{input-description-10}

You'll be given two words or sets of words separated by a question mark.
Your task is to replace the question mark with information about the
validity of the anagram. Example:

\begin{verbatim}
"Clint Eastwood" ? "Old West Action"
"parliament" ? "partial man"
\end{verbatim}

\subsection{Output Description}\label{output-description-10}

You should replace the question mark with some marker about the validity
of the anagram proposed. Example:

\begin{verbatim}
"Clint Eastwood" is an anagram of "Old West Action"
"parliament" is NOT an anagram of "partial man"
\end{verbatim}

\subsection{Challenge Input}\label{challenge-input-10}

\begin{verbatim}
"wisdom" ? "mid sow"
"Seth Rogan" ? "Gathers No"
"Reddit" ? "Eat Dirt"
"Schoolmaster" ? "The classroom"
"Astronomers" ? "Moon starer"
"Vacation Times" ? "I'm Not as Active"
"Dormitory" ? "Dirty Rooms"
\end{verbatim}

\subsection{Challenge Output}\label{challenge-output-8}

\begin{verbatim}
"wisdom" is an anagram of "mid sow"
"Seth Rogan" is an anagram of "Gathers No"
"Reddit" is NOT an anagram of "Eat Dirt"
"Schoolmaster" is an anagram of "The classroom"
"Astronomers" is NOT an anagram of "Moon starer"
"Vacation Times" is an anagram of "I'm Not as Active"
"Dormitory" is NOT an anagram of "Dirty Rooms"
\end{verbatim}

\subsection{Scala Solution}\label{scala-solution-10}

\begin{verbatim}
def anagram(s1:String, s2:String): Boolean = s1.toLowerCase().filter(_.isLetter).sorted == s2.toLowerCase().filter(_.isLetter).sorted
\end{verbatim}

\section{Title}\label{title-15}

Making Imgur-style Links

\subsection{Difficulty}\label{difficulty-14}

Easy

\subsection{Tags}\label{tags-15}

numbers, base62, shortened strings

\subsection{Description}\label{description-15}

Short links have been all the rage for several years now, spurred in
part by Twitter's character limits. Imgur - Reddit's go-to image hosting
site - uses a similar style for their links. Monotonically increasing
IDs represented in Base62.

Your task today is to convert a number to its Base62 representation.

\subsection{Input Description}\label{input-description-11}

You'll be given one number per line. Assume this is your alphabet:

\begin{verbatim}
0123456789abcdefghijklmnopqrstuvwxyzABCDEFGHIJKLMNOPQRSTUVWXYZ 
\end{verbatim}

Example input:

\begin{verbatim}
15674
7026425611433322325
\end{verbatim}

\subsection{Output Description}\label{output-description-11}

Your program should emit the number represented in Base62 notation.
Examples:

\begin{verbatim}
O44
bDcRfbr63n8
\end{verbatim}

\subsection{Challenge Input}\label{challenge-input-11}

\begin{verbatim}
187621
237860461
2187521
18752
\end{verbatim}

\subsection{Challenge Output}\label{challenge-output-9}

\begin{verbatim}
9OM
3n26g
B4b9
sS4    
\end{verbatim}

\subsection{Python Solution}\label{python-solution}

\begin{verbatim}
def toBase62(n):
    alphabet = "0123456789abcdefghijklmnopqrstuvwxyzABCDEFGHIJKLMNOPQRSTUVWXYZ"
    basis = len(alphabet)
    ret = ''
    while (n > 0):
        tmp = n % basis
        ret += alphabet[tmp]
        n = (n/basis)
    return ret
\end{verbatim}

\section{Title}\label{title-16}

Baum-Sweet Sequence

\subsection{Difficulty}\label{difficulty-15}

Easy

\subsection{Tags}\label{tags-16}

integer sequence, Baum-Sweet, infinite sequence, number theory

\subsection{Description}\label{description-16}

In mathematics, the
\href{https://en.wikipedia.org/wiki/Baum\%E2\%80\%93Sweet_sequence}{Baum--Sweet
sequence} is an infinite automatic sequence of 0s and 1s defined by the
rule:

\begin{itemize}
\itemsep1pt\parskip0pt\parsep0pt
\item
  b\_n = 1 if the binary representation of n contains no block of
  consecutive 0s of odd length;
\item
  b\_n = 0 otherwise;
\end{itemize}

for n \textgreater{}= 0.

For example, b\_4 = 1 because the binary representation of 4 is 100,
which only contains one block of consecutive 0s of length 2; whereas
b\_5 = 0 because the binary representation of 5 is 101, which contains a
block of consecutive 0s of length 1. When n is 19611206, b\_n is 0
because:

\begin{verbatim}
19611206 = 1001010110011111001000110 base 2
            00 0 0  00     00 000  0 runs of 0s
               ^ ^            ^^^    odd length sequences
           
\end{verbatim}

Because we find an odd length sequence of 0s, b\_n is 0.

\subsection{Challenge Description}\label{challenge-description}

Your challenge today is to write a program that generates the Baum-Sweet
sequence from 0 to some number \emph{n}. For example, given ``20'' your
program would emit:

\begin{verbatim}
1, 1, 0, 1, 1, 0, 0, 1, 0, 1, 0, 0, 1, 0, 0, 1, 1, 0, 0, 1, 0
\end{verbatim}

\subsection{Scala Solution}\label{scala-solution-11}

\begin{verbatim}
def b(n:Int): Int = {
    if (0 == n) {return 1}
    if (n.toBinaryString.split("1").filter(_!="").map(_.length%2 != 0).contains(true)) {return 0}
    else {return 1}
}

def baum_sweet(n:Int): IndexedSeq[Int] = (0 to n).map(b)
\end{verbatim}

\subsection{Go Solution}\label{go-solution}

\begin{verbatim}
package main

import (
    "fmt"
    "os"
    "strconv"
    "strings"
)

func baumSweet(s string) int {
    zeroes := strings.Split(s, "1")
    for _, zero := range zeroes {
        if (len(zero) > 0) && ((len(zero) % 2) == 1) {
            return 1
        }
    }
    return 0
}

func main() {
    num, _ := strconv.ParseInt(os.Args[1], 10, 32)

    for n := 0; n <= int(num); n++ {
        s := strconv.FormatInt(int64(n), 2)
        fmt.Printf("%d ", baumSweet(s))
    }
    fmt.Printf("\n")
}
\end{verbatim}

\section{Title}\label{title-17}

Gold and Treasure: The Beale Cipher

\subsection{Difficulty}\label{difficulty-16}

Easy

\subsection{Tags}\label{tags-17}

cipher, encryption, decryption

\subsection{Description}\label{description-17}

In 1885, an author named James B. Ward published a pamphlet telling of a
long-lost treasure available to anyone clever enough to solve the puzzle
associated with it. Ward reported that around 1817, a man named Thomas
Jefferson Beale stumbled upon gold and silver deposits in what is now
Colorado. Agreeing to keep it all a secret, Beale's team had spent the
better part of two years quietly mining, then had taken the metals to
Virginia by wagon and buried them in a vault underground between 1819
and 1821. Beale had written three notes explaining where the treasure
was and who had legal rights to shares in it, encrypting each of these
using a different text.

Eventually, the second of the three texts was deciphered using a
slightly altered version of the Declaration of Independence. Each number
in the text corresponded to a word in the U.S. Declaration of
Independence. The first letter of each of those words spelled the
plaintext---with a few modifications for errors and spelling.

Your mission today is to go treasure hunting and to write a program to
decipher Beale's message.

\subsection{DECLARATION OF
INDEPENDENCE}\label{declaration-of-independence}

When in the course of human events it becomes necessary for one people
to dissolve the political bands which have connected them with another
and to assume among the powers of the earth the separate and equal
station to which the laws of nature and of nature's god entitle them a
decent respect to the opinions of mankind requires that they should
declare the causes which impel them to the separation we hold these
truths to be self evident that all men are created equal that they are
endowed by their creator with certain unalienable rights that among
these are life liberty and the pursuit of happiness that to secure these
rights governments are instituted among men deriving their just powers
from the consent of the governed that whenever any form of government
becomes destructive of these ends it is the right of the people to alter
or to abolish it and to institute new government laying its foundation
on such principles and organizing its powers in such form as to them
shall seem most likely to effect their safety and happiness prudence
indeed will dictate that governments long established should not be
changed for light and transient causes and accordingly all experience
hath shown that mankind are more disposed to suffer while evils are
sufferable than to right themselves by abolishing the forms to which
they are accustomed but when a long train of abuses and usurpations
pursuing invariably the same object evinces a design to reduce them
under absolute despotism it is their right it is their duty to throw off
such government and to provide new guards for their future security such
has been the patient sufferance of these colonies and such is now the
necessity which constrains them to alter their former systems of
government the history of the present king of great Britain is a history
of repeated injuries and usurpations all having in direct object the
establishment of an absolute tyranny over these states to prove this let
facts be submitted to a candid world he has refused his assent to laws
the most wholesome and necessary for the public good he has forbidden
his governors to pass laws of immediate and pressing importance unless
suspended in their operation till his assent should be obtained and when
so suspended he has utterly neglected to attend to them he has refused
to pass other laws for the accommodation of large districts of people
unless those people would relinquish the right of representation in the
legislature a right inestimable to them and formidable to tyrants only
he has called together legislative bodies at places unusual
uncomfortable and distant from the depository of their public records
for the sole purpose of fatiguing them into compliance with his measures
he has dissolved representative houses repeatedly for opposing with
manly firmness his invasions on the rights of the people he has refused
for a long time after such dissolutions to cause others to be elected
whereby the legislative powers incapable of annihilation have returned
to the people at large for their exercise the state remaining in the
meantime exposed to all the dangers of invasion from without and
convulsions within he has endeavored to prevent the population of these
states for that purpose obstructing the laws for naturalization of
foreigners refusing to pass others to encourage their migration hither
and raising the conditions of new appropriations of lands he has
obstructed the administration of justice by refusing his assent to laws
for establishing judiciary powers he has made judges dependent on his
will alone for the tenure of their offices and the amount and payment of
their salaries he has erected a multitude of new offices and sent hither
swarms of officers to harass our people and eat out their substance he
has kept among us in times of peace standing armies without the consent
of our legislatures he has affected to render the military independent
of and superior to the civil power he has combined with others to
subject us to a jurisdiction foreign to our constitution and
unacknowledged by our laws giving his assent to their acts of pretended
legislation for quartering large bodies of armed troops among us for
protecting them by a mock trial from punishment for any murders which
they should commit on the inhabitants of these states for cutting off
our trade with all parts of the world for imposing taxes on us without
our consent for depriving us in many cases of the benefits of trial by
jury for transporting us beyond seas to be tried for pretended offenses
for abolishing the free system of English laws in a neighboring province
establishing therein an arbitrary government and enlarging its
boundaries so as to render it at once an example and fit instrument for
introducing the same absolute rule into these colonies for taking away
our charters abolishing our most valuable laws and altering
fundamentally the forms of our governments for suspending our own
legislature and declaring themselves invested with power to legislate
for us in all cases whatsoever he has abdicated government here by
declaring us out of his protection and waging war against us he has
plundered our seas ravaged our coasts burnt our towns and destroyed the
lives of our people he is at this time transporting large armies of
foreign mercenaries to complete the works of death desolation and
tyranny already begun with circumstances of cruelty and perfidy scarcely
paralleled in the most barbarous ages and totally unworthy the head of a
civilized nation he has constrained our fellow citizens taken captive on
the high seas to bear arms against their country to become the
executioners of their friends and brethren or to fall themselves by
their hands he has excited domestic insurrections amongst us and has
endeavored to bring on the inhabitants of our frontiers the merciless
Indian savages whose known rule of warfare is an undistinguished
destruction of all ages sexes and conditions in every stage of these
oppressions we have petitioned for redress in the most humble terms our
repeated petitions have been answered only by repeated injury a prince
whole character is thus marked by every act which may define a tyrant is
unfit to be the ruler of a free people nor have we been wanting in
attention to our British brethren we have warned them from time to time
of attempts by their legislature to extend an unwarrantable jurisdiction
over us we have reminded them of the circumstances of our emigration and
settlement here we have appealed to their native justice and magnanimity
and we have conjured them by the ties of our common kindred to disavow
these usurpations which would inevitably interrupt our connections and
correspondence they too have been deaf to the voice of justice and of
consanguinity we must therefore acquiesce in the necessity which
denounces our separation and hold them as we hold the rest of mankind
enemies in war in peace friends we therefore the representatives of the
united states of America in general congress assembled appealing to the
supreme judge of the world for the rectitude of our intentions do in the
name and by authority of the good people of these colonies solemnly
publish and declare that these united colonies are and of right ought to
be free and independent states that they are absolved from all
allegiance to the British crown and that all political connection
between them and the state of great Britain is and ought to be totally
dissolved and that as free and independent states they have full power
to levy war conclude peace contract alliances establish commerce and to
do all other acts and things which independent states may of right do
and for the support of this declaration with a firm reliance on the
protection of divine providence we mutually pledge to each other our
lives our fortunes and our sacred honor .

\subsection{Input Description}\label{input-description-12}

You'll be given a list of numbers, comma separated, representing the
ciphertext given by Beale. Example:

115, 73, 24, 807, 37, 52, 49, 17, 31, 62, 647, 22, 7, 15, 140, 47, 29,
107, 79, 84, 56, 239, 10, 26, 811, 5, 196, 308, 85, 52, 160, 136, 59,
211, 36, 9, 46, 316, 554, 122, 106, 95, 53, 58, 2, 42, 7, 35, 122, 53,
31, 82, 77, 250, 196, 56, 96, 118, 71, 140, 287, 28, 353, 37, 1005, 65,
147, 807, 24, 3, 8, 12, 47, 43, 59, 807, 45, 316, 101, 41, 78, 154,
1005, 122, 138, 191, 16, 77, 49, 102, 57, 72, 34, 73, 85, 35, 371, 59,
196, 81, 92, 191, 106, 273, 60, 394, 620, 270, 220, 106, 388, 287, 63,
3, 6, 191, 122, 43, 234, 400, 106, 290, 314, 47, 48, 81, 96, 26, 115,
92, 158, 191, 110, 77, 85, 197, 46, 10, 113, 140, 353, 48, 120, 106, 2,
607, 61, 420, 811, 29, 125, 14, 20, 37, 105, 28, 248, 16, 159, 7, 35,
19, 301, 125, 110, 486, 287, 98, 117, 511, 62, 51, 220, 37, 113, 140,
807, 138, 540, 8, 44, 287, 388, 117, 18, 79, 344, 34, 20, 59, 511, 548,
107, 603, 220, 7, 66, 154, 41, 20, 50, 6, 575, 122, 154, 248, 110, 61,
52, 33, 30, 5, 38, 8, 14, 84, 57, 540, 217, 115, 71, 29, 84, 63, 43,
131, 29, 138, 47, 73, 239, 540, 52, 53, 79, 118, 51, 44, 63, 196, 12,
239, 112, 3, 49, 79, 353, 105, 56, 371, 557, 211, 505, 125, 360, 133,
143, 101, 15, 284, 540, 252, 14, 205, 140, 344, 26, 811, 138, 115, 48,
73, 34, 205, 316, 607, 63, 220, 7, 52, 150, 44, 52, 16, 40, 37, 158,
807, 37, 121, 12, 95, 10, 15, 35, 12, 131, 62, 115, 102, 807, 49, 53,
135, 138, 30, 31, 62, 67, 41, 85, 63, 10, 106, 807, 138, 8, 113, 20, 32,
33, 37, 353, 287, 140, 47, 85, 50, 37, 49, 47, 64, 6, 7, 71, 33, 4, 43,
47, 63, 1, 27, 600, 208, 230, 15, 191, 246, 85, 94, 511, 2, 270, 20, 39,
7, 33, 44, 22, 40, 7, 10, 3, 811, 106, 44, 486, 230, 353, 211, 200, 31,
10, 38, 140, 297, 61, 603, 320, 302, 666, 287, 2, 44, 33, 32, 511, 548,
10, 6, 250, 557, 246, 53, 37, 52, 83, 47, 320, 38, 33, 807, 7, 44, 30,
31, 250, 10, 15, 35, 106, 160, 113, 31, 102, 406, 230, 540, 320, 29, 66,
33, 101, 807, 138, 301, 316, 353, 320, 220, 37, 52, 28, 540, 320, 33, 8,
48, 107, 50, 811, 7, 2, 113, 73, 16, 125, 11, 110, 67, 102, 807, 33, 59,
81, 158, 38, 43, 581, 138, 19, 85, 400, 38, 43, 77, 14, 27, 8, 47, 138,
63, 140, 44, 35, 22, 177, 106, 250, 314, 217, 2, 10, 7, 1005, 4, 20, 25,
44, 48, 7, 26, 46, 110, 230, 807, 191, 34, 112, 147, 44, 110, 121, 125,
96, 41, 51, 50, 140, 56, 47, 152, 540, 63, 807, 28, 42, 250, 138, 582,
98, 643, 32, 107, 140, 112, 26, 85, 138, 540, 53, 20, 125, 371, 38, 36,
10, 52, 118, 136, 102, 420, 150, 112, 71, 14, 20, 7, 24, 18, 12, 807,
37, 67, 110, 62, 33, 21, 95, 220, 511, 102, 811, 30, 83, 84, 305, 620,
15, 2, 10, 8, 220, 106, 353, 105, 106, 60, 275, 72, 8, 50, 205, 185,
112, 125, 540, 65, 106, 807, 138, 96, 110, 16, 73, 33, 807, 150, 409,
400, 50, 154, 285, 96, 106, 316, 270, 205, 101, 811, 400, 8, 44, 37, 52,
40, 241, 34, 205, 38, 16, 46, 47, 85, 24, 44, 15, 64, 73, 138, 807, 85,
78, 110, 33, 420, 505, 53, 37, 38, 22, 31, 10, 110, 106, 101, 140, 15,
38, 3, 5, 44, 7, 98, 287, 135, 150, 96, 33, 84, 125, 807, 191, 96, 511,
118, 40, 370, 643, 466, 106, 41, 107, 603, 220, 275, 30, 150, 105, 49,
53, 287, 250, 208, 134, 7, 53, 12, 47, 85, 63, 138, 110, 21, 112, 140,
485, 486, 505, 14, 73, 84, 575, 1005, 150, 200, 16, 42, 5, 4, 25, 42, 8,
16, 811, 125, 160, 32, 205, 603, 807, 81, 96, 405, 41, 600, 136, 14, 20,
28, 26, 353, 302, 246, 8, 131, 160, 140, 84, 440, 42, 16, 811, 40, 67,
101, 102, 194, 138, 205, 51, 63, 241, 540, 122, 8, 10, 63, 140, 47, 48,
140, 288

\subsection{Output Description}\label{output-description-12}

Your program should consume the input and decrypt it. Remember - the
first letter of that word number from the US Declaration of
Independence. Spacing, punctuation, capitalization, and fixing spelling
is left as an exercise to the treasure seeker (as Beale intended). The
above letter was intended to decrypt to:

\emph{I have deposited in the county of Bedford, about four miles from
Buford's, in an excavation or vault, six feet below the surface of the
ground, the following articles, belonging jointly to the parties whose
names are given in number ``3,'' herewith:}

\emph{The first deposit consisted of one thousand and fourteen pounds of
gold, and three thousand eight hundred and twelve pounds of silver,
deposited November, 1819. The second was made December, 1821, and
consisted of nineteen hundred and seven pounds of gold, and twelve
hundred and eighty-eight pounds of silver; also jewels, obtained in
St.~Louis in exchange for silver to save transportation, and valued at
\$13,000.}

\emph{The above is securely packed in iron pots, with iron covers. The
vault is roughly lined with stone, and the vessels rest on solid stone,
and are covered with others. Paper number ``1'' describes the exact
locality of the vault so that no difficulty will be had in finding it.}

\subsection{Challenge Input}\label{challenge-input-12}

71, 194, 38, 1701, 89, 76, 11, 83, 1629, 48, 94, 63, 132, 16, 111, 95,
84, 341, 975, 14, 40, 64, 27, 81, 139, 213, 63, 90, 1120, 8, 15, 3, 126,
2018, 40, 74, 758, 485, 604, 230, 436, 664, 582, 150, 251, 284, 308,
231, 124, 211, 486, 225, 401, 370, 11, 101, 305, 139, 189, 17, 33, 88,
208, 193, 145, 1, 94, 73, 416, 918, 263, 28, 500, 538, 356, 117, 136,
219, 27, 176, 130, 10, 460, 25, 485, 18, 436, 65, 84, 200, 283, 118,
320, 138, 36, 416, 280, 15, 71, 224, 961, 44, 16, 401, 39, 88, 61, 304,
12, 21, 24, 283, 134, 92, 63, 246, 486, 682, 7, 219, 184, 360, 780, 18,
64, 463, 474, 131, 160, 79, 73, 440, 95, 18, 64, 581, 34, 69, 128, 367,
460, 17, 81, 12, 103, 820, 62, 116, 97, 103, 862, 70, 60, 1317, 471,
540, 208, 121, 890, 346, 36, 150, 59, 568, 614, 13, 120, 63, 219, 812,
2160, 1780, 99, 35, 18, 21, 136, 872, 15, 28, 170, 88, 4, 30, 44, 112,
18, 147, 436, 195, 320, 37, 122, 113, 6, 140, 8, 120, 305, 42, 58, 461,
44, 106, 301, 13, 408, 680, 93, 86, 116, 530, 82, 568, 9, 102, 38, 416,
89, 71, 216, 728, 965, 818, 2, 38, 121, 195, 14, 326, 148, 234, 18, 55,
131, 234, 361, 824, 5, 81, 623, 48, 961, 19, 26, 33, 10, 1101, 365, 92,
88, 181, 275, 346, 201, 206, 86, 36, 219, 324, 829, 840, 64, 326, 19,
48, 122, 85, 216, 284, 919, 861, 326, 985, 233, 64, 68, 232, 431, 960,
50, 29, 81, 216, 321, 603, 14, 612, 81, 360, 36, 51, 62, 194, 78, 60,
200, 314, 676, 112, 4, 28, 18, 61, 136, 247, 819, 921, 1060, 464, 895,
10, 6, 66, 119, 38, 41, 49, 602, 423, 962, 302, 294, 875, 78, 14, 23,
111, 109, 62, 31, 501, 823, 216, 280, 34, 24, 150, 1000, 162, 286, 19,
21, 17, 340, 19, 242, 31, 86, 234, 140, 607, 115, 33, 191, 67, 104, 86,
52, 88, 16, 80, 121, 67, 95, 122, 216, 548, 96, 11, 201, 77, 364, 218,
65, 667, 890, 236, 154, 211, 10, 98, 34, 119, 56, 216, 119, 71, 218,
1164, 1496, 1817, 51, 39, 210, 36, 3, 19, 540, 232, 22, 141, 617, 84,
290, 80, 46, 207, 411, 150, 29, 38, 46, 172, 85, 194, 39, 261, 543, 897,
624, 18, 212, 416, 127, 931, 19, 4, 63, 96, 12, 101, 418, 16, 140, 230,
460, 538, 19, 27, 88, 612, 1431, 90, 716, 275, 74, 83, 11, 426, 89, 72,
84, 1300, 1706, 814, 221, 132, 40, 102, 34, 868, 975, 1101, 84, 16, 79,
23, 16, 81, 122, 324, 403, 912, 227, 936, 447, 55, 86, 34, 43, 212, 107,
96, 314, 264, 1065, 323, 428, 601, 203, 124, 95, 216, 814, 2906, 654,
820, 2, 301, 112, 176, 213, 71, 87, 96, 202, 35, 10, 2, 41, 17, 84, 221,
736, 820, 214, 11, 60, 760

\subsection{Note}\label{note-1}

The inspiration for this challenge comes from the
\href{http://www.damninteresting.com/89-263-201-500-337-480/}{Damn
Interesting website} and my love of the Nicholas Cage movie series
``National Treasure''. For more info see the
\href{http://www.unmuseum.org/bealepap.htm}{Museum of Unnatural
History}.

\subsection{Python Solution}\label{python-solution-1}

\begin{verbatim}
def solution(msg):
    # msg is a comma-separated list of integers, just like beale wrote out
    decl = "When in the course of human events ..." # omitted for brevity
    msg = map(int, map(str.strip, msg1.split(',')))
    return ''.join([ decl.split()[x-1][0] for x in msg])
\end{verbatim}

\section{Title}\label{title-18}

Capitalize The First Letter of Every Word

\subsection{Difficulty}\label{difficulty-17}

Easy

\subsection{Tags}\label{tags-18}

words

\subsection{Description}\label{description-18}

Given a sentence, can you capitalize the first letter of every word?

Yes this is a built-in in Python (\texttt{string.capwords}) and maybe
some other languages, but the challenge is to implement your own
\texttt{capwords}-like method.

\subsection{Input Description}\label{input-description-13}

You'll be given an Enlish language sentence like this:

\begin{verbatim}
Now is the time for all great programmers to capitalize the correct words.
\end{verbatim}

\subsection{Output Description}\label{output-description-13}

You should emit a sentence with the first letter of every word
capitalized.

\begin{verbatim}
Now Is The Time For All Great Programmers To Capitalize The Correct Words.
\end{verbatim}

\subsection{Challenge Input}\label{challenge-input-13}

\begin{verbatim}
Education is an admirable thing, but it is well to remember from time to time that nothing that is worth knowing can be taught.
An intelligent man is sometimes forced to be drunk to spend time with his fools.
The heart of a mother is a deep abyss at the bottom of which you will always find forgiveness.
All things are subject to interpretation whichever interpretation prevails at a given time is a function of power and not truth.
\end{verbatim}

\subsection{Challenge Output}\label{challenge-output-10}

\begin{verbatim}
Education Is An Admirable Thing, But It Is Well To Remember From Time To Time That Nothing That Is Worth Knowing Can Be Taught.
An Intelligent Man Is Sometimes Forced To Be Drunk To Spend Time With His Fools.
The Heart Of A Mother Is A Deep Abyss At The Bottom Of Which You Will Always Find Forgiveness.
All Things Are Subject To Interpretation Whichever Interpretation Prevails At A Given Time Is A Function Of Power And Not Truth.
\end{verbatim}

\subsection{Scala Solution}\label{scala-solution-12}

\begin{verbatim}
def capwords(s:String) = s.split(" ").map(_.capitalize).mkString(" ")
\end{verbatim}

\section{Title}\label{title-19}

Collatz Conjecture

\subsection{Difficulty}\label{difficulty-18}

Easy

\subsection{Tags}\label{tags-19}

number theory, infinite sequence, integer sequence

\subsection{Description}\label{description-19}

The \href{https://en.wikipedia.org/wiki/Collatz_conjecture}{Collatz
conjecture} is a conjecture in mathematics named after Lothar Collatz,
who first proposed it in 1937. Take any natural number n. If n is even,
divide it by 2 to get n / 2. If n is odd, multiply it by 3 and add 1 to
obtain 3n + 1. Repeat the process (which has been called ``Half Or
Triple Plus One'', or HOTPO{[}6{]}) indefinitely. The conjecture is that
no matter what number you start with, you will always eventually reach
1.

For instance, starting with n = 6, one gets the sequence 6, 3, 10, 5,
16, 8, 4, 2, 1. n = 19, for example, takes longer to reach 1: 19, 58,
29, 88, 44, 22, 11, 34, 17, 52, 26, 13, 40, 20, 10, 5, 16, 8, 4, 2, 1.

\subsection{Input}\label{input-1}

Your program should take a positive integer \emph{N} as an argument.

\subsection{Output Description}\label{output-description-14}

Your program should emit the number of steps it takes to reach 1 and the
sequence emitted.

\subsection{Bonus 1}\label{bonus-1}

Can you explain what's so interesting about the number 9232 in the
context of the Collatz Conjecture?

\subsection{Bonus 2}\label{bonus-2}

If you're feeling like it, throw in some unique visualizations of
sequences or series from the Collatz Conjecture, using any imaging
library you wish.

\subsection{Scala Solution}\label{scala-solution-13}

\begin{verbatim}
def collatz(N:Int): List[Int] = {
  def loop(n:Int, sofar:List[Int]): List[Int] = {
    n match {
      case 1 => 1::sofar
      case _ => (n%2 == 0) match {
          case true => loop(n/2, n::sofar)
          case false => loop(1 + 3*n, n::sofar)
      }
    }
  }
  loop(N, List()).reverse
}
\end{verbatim}

\section{Title}\label{title-20}

Concatenated Integers

\subsection{Difficulty}\label{difficulty-19}

Easy

\subsection{Tags}\label{tags-20}

math, concatenation

\subsection{Description}\label{description-20}

Given a list of integers separated by a single space on standard input,
print out the largest and smallest values that can be obtained by
concatenating the integers together on their own line. This is from
\href{https://blog.svpino.com/2015/05/07/five-programming-problems-every-software-engineer-should-be-able-to-solve-in-less-than-1-hour}{Five
programming problems every Software Engineer should be able to solve in
less than 1 hour}, problem 4. Leading 0s are not allowed (e.g.~01234 is
not a valid entry).

\subsection{Sample Input}\label{sample-input-1}

You'll be given a handful of integers per line. Example:

\begin{verbatim}
5 56 50
\end{verbatim}

\subsection{Sample Output}\label{sample-output-1}

You should emit the smallest and largest integer you can make, per line.
Example:

\begin{verbatim}
50556 56550
\end{verbatim}

\subsection{Challenge Input}\label{challenge-input-14}

\begin{verbatim}
79 82 34 83 69
420 34 19 71 341
17 32 91 7 46
\end{verbatim}

\subsection{Challenge Output}\label{challenge-output-11}

\begin{verbatim}
3469798283 8382796934
193413442071 714203434119
173246791 917463217
\end{verbatim}

\subsection{Scala Solution}\label{scala-solution-14}

\begin{verbatim}
// returns min, max
def intConcat(s:String): (Long, Long) = {
    val l = s.split(" ").permutations.map(_.mkString.toLong).toList
    (l.sorted.head, l.sorted.reverse.head)
}
\end{verbatim}

\section{Title}\label{title-21}

Condensing Sentences

\subsection{Difficulty}\label{difficulty-20}

Easy

\subsection{Tags}\label{tags-21}

word games, word play

\subsection{Description}\label{description-21}

Compression makes use of the fact that repeated structures are
redundant, and it's more efficient to represent the pattern and the
count or a reference to it. Siimilarly, we can \emph{condense} a
sentence by using the redundancy of overlapping letters from the end of
one word and the start of the next. In this manner we can reduce the
size of the sentence, even if we start to lose meaning.

For instance, the phrase ``live verses'' can be condensed to
``liverses''.

In this challenge you'll be asked to write a tool to condense sentences.

\subsection{Input Description}\label{input-description-14}

You'll be given a sentence, one per line, to condense. Condense where
you can, but know that you can't condense everywhere. Example:

\begin{verbatim}
I heard the pastor sing live verses easily.
\end{verbatim}

\subsection{Output Description}\label{output-description-15}

Your program should emit a sentence with the appropriate parts condensed
away. Our example:

\begin{verbatim}
I heard the pastor sing liverses easily. 
\end{verbatim}

\subsection{Challenge Input}\label{challenge-input-15}

\begin{verbatim}
Deep episodes of Deep Space Nine came on the television only after the news.
Digital alarm clocks scare area children.
\end{verbatim}

\subsection{Challenge Output}\label{challenge-output-12}

\begin{verbatim}
Deep episodes of Deep Space Nine came on the televisionly after the news.
Digitalarm clockscarea children.
\end{verbatim}

\section{Title}\label{title-22}

Double (or More) Knots

\subsection{Difficulty}\label{difficulty-21}

Easy

\subsection{Tags}\label{tags-22}

artwork, ASCII art

\subsection{Description}\label{description-22}

In knot tying, the double overhand knot is a common, basic knot that
serves fishermen (and fisher-women) and sourgeons well. We can represent
a single order double knot like this:

\begin{verbatim}
  __
 /  \
| /\ |
| \/ |
 \ \/
 /\ \
| /\ |
| \/ |
 \__/
\end{verbatim}

And a double knot of 2 this way:

\begin{verbatim}
  __  __ 
 /  \/  \
| /\/ /\ |
| \/ /\/ |
 \ \/\ \/
 /\ \/\ \
| /\/ /\ |
| \/ /\/ |
 \__/\__/
\end{verbatim}

\subsection{Challenge Description}\label{challenge-description-1}

In this challenge, you'll be asked to draw various orders of ASCII
double knots given an integer.

\subsection{Challenge Input}\label{challenge-input-16}

\begin{verbatim}
2
6
\end{verbatim}

\subsection{Challenge Output}\label{challenge-output-13}

\begin{verbatim}
  __  __ 
 /  \/  \
| /\/ /\ |
| \/ /\/ |
 \ \/\ \/
 /\ \/\ \
| /\/ /\ |
| \/ /\/ |
 \__/\__/

   __  __  __  __  __  __
  /  \/  \/  \/  \/  \/  \
 | /\/ /\/ /\/ /\/ /\/ /\ |
 | \/ /\/ /\/ /\/ /\/ /\/ |
  \ \/\ \/\ \/\ \/\ \/\ \/
  /\ \/\ \/\ \/\ \/\ \/\ \
 | /\/ /\/ /\/ /\/ /\/ /\ |
 | \/ /\/ /\/ /\/ /\/ /\/ |
  \__/\__/\__/\__/\__/\__/
\end{verbatim}

\section{Title}\label{title-23}

Emirp Numbers

\subsection{Difficulty}\label{difficulty-22}

Easy

\subsection{Tags}\label{tags-23}

math, number theory, prime numbers, integer sequence

\subsection{Description}\label{description-23}

We all know what prime numbers are - numbers only divisible by
themselves and one. Math grad student Matthew Scroggs came up with
\emph{emirp} numbers. An
\href{https://en.wikipedia.org/wiki/Emirp}{emirp number} is a prime
number which is a different prime number when the digits are reversed.

Your task today is to write a program that can discover emirp numbers
over a range.

\subsection{Input Description}\label{input-description-15}

You'll be given two numbers, \emph{a} and \emph{b}, that represent the
range \emph{inclusive} of the numbers to screen for emirp numbers.
Example:

\begin{verbatim}
10 100
\end{verbatim}

\subsection{Output Description}\label{output-description-16}

Your program should emit the list of valid emirp numbers for that range.
Example:

\begin{verbatim}
[11; 13; 17; 31; 37; 71; 73; 79; 97]
\end{verbatim}

\subsection{Challenge Input}\label{challenge-input-17}

\begin{verbatim}
10000 10100
999810 999999
\end{verbatim}

\subsection{Challenge Output}\label{challenge-output-14}

\begin{verbatim}
[10007; 10009; 10039; 10061; 10067; 10069; 10079; 10091]
[999853; 999931; 999983]
\end{verbatim}

\subsection{FSharp Solution}\label{fsharp-solution-1}

\begin{verbatim}
let isprime (n:int) =
    let rec check i =
        i > n/2 || (n % i <> 0 && check (i + 1))
    check 2;;

let revnum(n:int) = 
    string(n).ToCharArray() |> Array.rev |> Array.map(fun x -> string(x)) |> String.concat "" |> int


[ 1 .. 100 ] |> List.filter isprime |> List.filter (fun x -> revnum x |> isprime)
\end{verbatim}

\section{Title}\label{title-24}

Extravagant Numbers

\subsection{Difficulty}\label{difficulty-23}

Easy

\subsection{Tags}\label{tags-24}

number theory, math, integer sequence

\subsection{Description}\label{description-24}

An \href{https://en.wikipedia.org/wiki/Extravagant_number}{extravagant
number} (also known as a wasteful number) is a natural number that has
fewer digits than the number of digits in its prime factorization
(including exponents). Trivial examples include 6 = 2*3, 8 = 2\^{}3, and
9 = 3\^{}2, all extravagant numbers.

Your challenge today is to write a program to determine is numbers are
extravagant or not.

\subsection{Input Description}\label{input-description-16}

You'll be given a single integer per line. Examples:

\begin{verbatim}
6
16
32
99
\end{verbatim}

\subsection{Output Description}\label{output-description-17}

Your program should emit if the number is extravagant or not:

\begin{verbatim}
6 EXTRAVAGANT
16 NOT EXTRAVAGANT
32 NOT EXTRAVAGANT
99 EXTRAVAGANT
\end{verbatim}

\subsection{Challenge Input}\label{challenge-input-18}

\begin{verbatim}
90
30
74
141
782
938
\end{verbatim}

\subsection{Challenge Output}\label{challenge-output-15}

\begin{verbatim}
90 EXTRAVAGANT
30 EXTRAVAGANT
74 EXTRAVAGANT
141 NOT EXTRAVAGANT
782 EXTRAVAGANT
938 EXTRAVAGANT
\end{verbatim}

\subsection{Scala Solution}\label{scala-solution-15}

\begin{verbatim}
def factorize(x: Int): List[Int] = {
    @tailrec
    def foo(x: Int, a: Int = 2, list: List[Int] = Nil): List[Int] = a*a > x match {
      case false if x % a == 0 => foo(x / a, a    , a :: list)
      case false               => foo(x    , a + 1, list)
      case true                => x :: list
    }
    foo(x)
  }

def power(n:Int): String = {
    if (n > 1) {return n.toString}
    else {return ""}
}

def extravagant(n:Int): Boolean = 
    factorize(n).groupBy(x => x).mapValues(_.length).map(x => x._1 + "" + power(x._2)).mkString("").length > n.toString.length
\end{verbatim}

\section{Title}\label{title-25}

Finding the Freshest Eggs

\subsection{Difficulty}\label{difficulty-24}

Easy

\subsection{Tags}\label{tags-25}

dates

\subsection{Description}\label{description-25}

The fresher the egg, the better the flavor. Because the sell-by date for
eggs in a supermarket (with U.S.D.A. inspection) can be up to 45 days
after the packing date, there is a quick and easy way to check for
freshness: the Julian date. Every egg carton has a code printed on its
side, and the last 3 digits of this code are called the Julian date. The
code uses a number from 001 to 365 to correspond to a day of the year
and indicate when they were packaged. For example, 001 is January 1st
and 365 is December 31st.

For this challenge, your goal is to read the ISO calendar date
(e.g.~2015-03-28) and then the Julian date code from a number of egg
cartons and to pick the freshest one, the one that was packaged the
fewest days ago.

\subsection{Input Description}\label{input-description-17}

For each scenario you'll be given an integer \emph{N} which tells you
how many egg cartons to check, then on the next line an ISO date format
of the current date. Then you'll be given \emph{N} egg cartons' worth of
Julian dates.

You'll be given 3 sets of scenarios that match these parameters. Solve
them all independently.

You should work with the assumption that time travel is impossible, so
keep that in mind working with Julian dates.

\subsection{Output Description}\label{output-description-18}

Your program should emit the scenario's date and the Julian date of the
freshest egg carton and the oldest egg carton (maybe ripe for removal
from the shelf!).

\subsection{Challenge Input}\label{challenge-input-19}

Scenario 1:

\begin{verbatim}
5
2015-03-28
019 
026 
017 
041 
063 
\end{verbatim}

Scenario 2:

\begin{verbatim}
5 
2014-01-01
311 
163 
270 
229 
162 
\end{verbatim}

Scenario 3:

\begin{verbatim}
5
2015-01-10
321 
004
354
009 
337 
\end{verbatim}

\subsection{Challenge Output}\label{challenge-output-16}

\begin{verbatim}
2015-03-28 063 017
2014-01-01 311 162
2015-01-10 009 321
\end{verbatim}

\section{Title}\label{title-26}

Basic Graph Statistics: Node Degrees

\subsection{Difficulty}\label{difficulty-25}

Easy

\subsection{Tags}\label{tags-26}

graph theory, graph theory!node, graph theory!edge, graph
theory!adjacency matrix

\subsection{Description}\label{description-26}

In graph theory, the \emph{degree} of a node is the number of edges
coming into it or going out of it - how connected it is. For this
challenge you'll be calculating the degree of every node.

\subsection{Input Description}\label{input-description-18}

First you'll be given an integer, \emph{N}, on one line showing you how
many nodes to account for. Next you'll be given an undirected graph as a
series of number pairs, \emph{a} and \emph{b}, showing that those two
nodes are connected - an edge. Example:

\begin{verbatim}
3 
1 2
1 3
\end{verbatim}

\subsection{Output Description}\label{output-description-19}

Your program should emit the degree for each node. Example:

\begin{verbatim}
Node 1 has a degree of 2
Node 2 has a degree of 1
Node 3 has a degree of 1
\end{verbatim}

\subsection{Challenge Input}\label{challenge-input-20}

This data set is an social network of tribes of the Gahuku-Gama alliance
structure of the Eastern Central Highlands of New Guinea, from Kenneth
Read (1954). The dataset contains a list of all of links, where a link
represents signed friendships between tribes. It was downloaded from
\href{http://networkrepository.com/soc_tribes.php}{the network
repository}.

\begin{verbatim}
16
1 2
1 3
2 3
1 4
3 4
1 5
2 5
1 6
2 6
3 6
3 7
5 7
6 7
3 8
4 8
6 8
7 8
2 9
5 9
6 9
2 10
9 10
6 11
7 11
8 11
9 11
10 11
1 12
6 12
7 12
8 12
11 12
6 13
7 13
9 13
10 13
11 13
5 14
8 14
12 14
13 14
1 15
2 15
5 15
9 15
10 15
11 15
12 15
13 15
1 16
2 16
5 16
6 16
11 16
12 16
13 16
14 16
15 16
\end{verbatim}

\subsection{Challenge Output}\label{challenge-output-17}

\begin{verbatim}
Node 1 has a degree of 8
Node 2 has a degree of 8
Node 3 has a degree of 6
Node 4 has a degree of 3
Node 5 has a degree of 7
Node 6 has a degree of 10
Node 7 has a degree of 7
Node 8 has a degree of 7
Node 9 has a degree of 7
Node 10 has a degree of 5
Node 11 has a degree of 9
Node 12 has a degree of 8
Node 13 has a degree of 8
Node 14 has a degree of 5
Node 15 has a degree of 9
Node 16 has a degree of 9
\end{verbatim}

\subsection{Bonus: Adjascency Matrix}\label{bonus-adjascency-matrix}

Another tool used in graph theory is an \emph{adjacency matrix}, which
is an \emph{N} by \emph{N} matrix where each \emph{(i,j)} cell is filled
out with the degree of connection between nodes \emph{i} and \emph{j}.
For our example graph above the adjacency matrix would look like this:

\begin{verbatim}
0 1 1
1 0 0
1 0 0
\end{verbatim}

Indicating that node 1 is connected to nodes 2 and 3, but nodes 2 and 3
do not connect. For a bonus, create the adjacency matrix for the
challenge graph.

\subsection{Scala Solution}\label{scala-solution-16}

\begin{verbatim}
def degree(edges:String) = 
    edges.
      split("\n").
      map(_.split(" ").filter(_.length>0)).
      toSet.
      toList.
      flatten.
      groupBy(_.toString).
      mapValues(_.size)

def adj_matrix(edges:String, n:Int):String = {
    val m = Array.ofDim[Int](n,n)
    val es = edges.
               split("\n").
               map(_.split(" ").filter(_.length>0)).
               map(_.map(_.toInt))
    for (e <- es) { m(e(0)-1)(e(1)-1) = 1; m(e(1)-1)(e(0)-1) = 1 }
    m.map(_.mkString(" ")).mkString("\n")
}

def challenge(edges:String) = 
    degree(edges).foreach { kv => println(kv._1 + " has a degree of " + kv._2) }

def bonus(edges:String, n:Int) = {
    challenge(edges)
    println(adj_matrix(edges, n))
}
\end{verbatim}

\subsection{Go Solution}\label{go-solution-1}

\begin{verbatim}
package main

import (
    "fmt"
    "io/ioutil"
    "os"
    "strconv"
    "strings"
)

func check(e error) {
    if e != nil {
        panic(e)
    }
}

func main() {
    bdata, err := ioutil.ReadFile(os.Args[1])
    check(err)

    data := string(bdata)
    var nodes map[string]int
    nodes = make(map[string]int)

    // calcuate node degree
    lines := strings.Split(data, "\n")
    for _, line := range lines {
        vals := strings.Split(line, " ")
        if len(vals) == 2 {
            nodes[vals[0]] = nodes[vals[0]] + 1
            nodes[vals[1]] = nodes[vals[1]] + 1
        }
    }
    i := 0
    for k, v := range nodes {
        fmt.Printf("Node %s has a degree of %d\n", k, v)
        i = i + 1
    }

    // bonus adjacency matrix
    adjm := make([][]string, i)
    for n := range adjm {
        adjm[n] = make([]string, i)
        for m := range adjm[n] {
            adjm[n][m] = "0"
        }
    }
    for _, line := range lines {
        vals := strings.Split(line, " ")
        if len(vals) == 2 {
            x, err := strconv.ParseUint(vals[0], 10, 32)
            check(err)
            y, err := strconv.ParseUint(vals[1], 10, 32)
            check(err)
            adjm[x-1][y-1] = "1"
            adjm[y-1][x-1] = "1"
            adjm[x-1][x-1] = "1"
        }
    }

    for n := 0; n < i; n++ {
        fmt.Printf("%q\n", strings.Join(adjm[n], " "))
    }
}
\end{verbatim}

\section{Title}\label{title-27}

Harshad Number

\subsection{Difficulty}\label{difficulty-26}

Easy

\subsection{Tags}\label{tags-27}

number theory, integer sequence, divisors

\subsection{Description}\label{description-27}

In recreational mathematics, a
\href{https://en.wikipedia.org/wiki/Harshad_number}{Harshad number} (or
Niven number) in a given number base, is an integer that is divisible by
the sum of its digits when written in that base. Harshad numbers in base
n are also known as \emph{n}-Harshad (or \emph{n}-Niven) numbers.
Harshad numbers were defined by D. R. Kaprekar, a mathematician from
India. The word ``Harshad'' comes from the Sanskrit \emph{harsa} (joy) +
\emph{da} (give), meaning joy-giver. The term ``Niven number'' arose
from a paper delivered by Ivan M. Niven at a conference on number theory
in 1977.

Thus, you can observe that number 21 is a Harshad number - 21 / (2 + 1)
= 7.

Today's challenge is to determine if an integer is a valid Harshad
number.

\subsection{Input Description}\label{input-description-19}

You'll be given an integer, one per line. Example:

\begin{verbatim}
21
22
\end{verbatim}

\subsection{Output Description}\label{output-description-20}

Your program should emit if the number is a Harshad number or not.
Example:

\begin{verbatim}
21 HARSHAD
22 NOT HARSHAD
\end{verbatim}

\subsection{Challenge Input}\label{challenge-input-21}

\begin{verbatim}
21
49
62
63
\end{verbatim}

\subsection{Challenge Output}\label{challenge-output-18}

\begin{verbatim}
21 HARSHAD 
49 NOT HARSHAD
62 NOT HARSHAD
63 HARSHAD
\end{verbatim}

\subsection{Scala Solution}\label{scala-solution-17}

\begin{verbatim}
def harshad(n:Int): Boolean = (n/n.toString.map(_.toString.toInt).sum)*(n.toString.map(_.toString.toInt).sum) == n
\end{verbatim}

\section{Title}\label{title-28}

Integer Sequence Search Part 1

\subsection{Difficulty}\label{difficulty-27}

Easy

\subsection{Tags}\label{tags-28}

integer sequence, math

\subsection{Description}\label{description-28}

In mathematics, an integer sequence is a sequence (i.e., an ordered
list) of integers. Not all sequences are computable (e.g.~not all have a
formula that can express them), but unique sequences have interesting
properties and can be quite fun to watch play out.

The \href{https://oeis.org/}{On-Line Encyclopedia of Integer Sequences
(OEIS)} website has an interesting feature where you can search for
sequences by name, ID, or even just subsequences.

For this challenge you'll be replicating that subsequence search
feature.

\subsection{Input Description}\label{input-description-20}

You'll be given two integers, \emph{N} and \emph{M}, which tell you how
many sequences to read to form your database and then how many search
queries to process, respectively. Then you'll be given the database as
\emph{N} pairs of \emph{name} and \emph{first 10 terms of the sequence}
pair. Then you'll be given \emph{M} queries of a series of integers.
Note that the overlap of the query and the sequence database will be
unambiguous but is not guaranteed to overlap completely. All sequences
to search will be contiguous (e.g.~no gaps). Sequence names will use the
OEIS naming convention

Example:

\begin{verbatim}
1 2
A000055 1, 1, 1, 1, 2, 3, 6, 11, 23, 47
11, 23, 47, 106
3, 14, 15, 92, 65
\end{verbatim}

\subsection{Output Description}\label{output-description-21}

For each of the search terms you should emit the query and the sequence
name.

Example:

\begin{verbatim}
11, 23, 47, 106   A000055
3, 14, 15, 92, 65 NOMATCH
\end{verbatim}

\subsection{Challenge Input}\label{challenge-input-22}

\begin{verbatim}
9 6
A000055 1, 1, 1, 1, 2, 3, 6, 11, 23, 47
A000045 0, 1, 1, 2, 3, 5, 8, 13, 21, 34
A050278 1023456789, 1023456798, 1023456879, 1023456897, 1023456978, 1023456987, 1023457689, 1023457698, 1023457869, 1023457896
A000010 1, 1, 2, 2, 4, 2, 6, 4, 6, 4
A194508 -1, 1, 0, 2, 1, 0, 2, 1, 3, 2
A000111 1, 1, 1, 2, 5, 16, 61, 272, 1385, 7936
A233586 1, 6, 12, 19, 63, 263, 856, 2632, 7714, 9683
A000391 1, 6, 21, 71, 216, 672, 1982, 5817, 16582, 46633
A000713 1, 3, 8, 18, 38, 74, 139, 249, 434, 734
1, 3, 8, 18
263, 856, 2632, 7714, 9683
1, 1, 2, 3, 5, 8
1, 6, 12, 19, 63
20, 22, 24, 28, 30, 34
434, 734, 1215, 1967, 3132, 4902
\end{verbatim}

\subsection{Challenge Output}\label{challenge-output-19}

\begin{verbatim}
1, 3, 8, 18     A000713
263, 856, 2632, 7714, 9683  A233586
1, 1, 2, 3, 5, 8    A000045
1, 6, 12, 19, 63    A233586
20, 22, 24, 28, 30, 34  NOMATCH
434, 734, 1215, 1967, 3132, 4902 A000713
\end{verbatim}

\section{Title}\label{title-29}

Jolly Jumper

\subsection{Difficulty}\label{difficulty-28}

Easy

\subsection{Tags}\label{tags-29}

word play

\subsection{Description}\label{description-29}

A sequence of n \textgreater{} 0 integers is called a jolly jumper if
the absolute values of the differences between successive elements take
on all possible values 1 through n - 1. For instance,

\begin{verbatim}
1 4 2 3
\end{verbatim}

is a jolly jumper, because the absolute differences are 3, 2, and 1,
respectively. The definition implies that any sequence of a single
integer is a jolly jumper. Write a program to determine whether each of
a number of sequences is a jolly jumper.

\subsection{Input Description}\label{input-description-21}

You'll be given a row of numbers. The first number tells you the number
of integers to calculate over, \emph{N}, followed by \emph{N} integers
to calculate the differences. Example:

\begin{verbatim}
4 1 4 2 3
8 1 6 -1 8 9 5 2 7
\end{verbatim}

\subsection{Output Description}\label{output-description-22}

Your program should emit some indication if the sequence is a jolly
jumper or not. Example:

\begin{verbatim}
4 1 4 2 3 JOLLY
8 1 6 -1 8 9 5 2 7 NOT JOLLY
\end{verbatim}

\subsection{Challenge Input}\label{challenge-input-23}

\begin{verbatim}
4 1 4 2 3
5 1 4 2 -1 6
4 19 22 24 21
4 19 22 24 25
4 2 -1 0 2
\end{verbatim}

\subsection{Challenge Output}\label{challenge-output-20}

\begin{verbatim}
4 1 4 2 3 JOLLY
5 1 4 2 -1 6 NOT JOLLY
4 19 22 24 21 NOT JOLLY
4 19 22 24 25 JOLLY
4 2 -1 0 2 JOLLY
\end{verbatim}

\section{Title}\label{title-30}

L33tspeak Translator

\subsection{Difficulty}\label{difficulty-29}

Easy

\subsection{Tags}\label{tags-30}

word play

\subsection{Description}\label{description-30}

L33tspeak - the act of speaking like a computer hacker (or hax0r) - was
popularized in the late 1990s as a mechanism of abusing ASCII art and
character mappings to confuse outsiders. It was a lot of fun.
\href{http://megatokyo.com/strip/9}{One popuar comic strip} in 2000
showed just how far the joke ran.

In L33Tspeak you substitute letters for their rough outlines in ASCII
characters, e.g.~symbols or numbers. You can have 1:1 mappings (like E
-\textgreater{} 3) or 1:many mappings (like W -\textgreater{} `//). So
then you wind up with words like this:

\begin{verbatim}
BASIC => 6451C
ELEET => 31337 (pronounced elite)
WOW => `//0`//
MOM => (V)0(V)
\end{verbatim}

\subsection{Mappings}\label{mappings}

For this challenge we'll be using a subset of American Standard
Leetspeak:

\begin{verbatim}
A -> 4
B -> 6
E -> 3
I -> 1
L -> 1
M -> (V)
N -> (\)
O -> 0
S -> 5
T -> 7
V -> \/
W -> `//
\end{verbatim}

Your challenge, should you choose to accept it, is to translate to and
from L33T.

\subsection{Input Description}\label{input-description-22}

You'll be given a word or a short phrase, one per line, and asked to
convert it from L33T or to L33T. Examples:

\begin{verbatim}
31337 
storm 
\end{verbatim}

\subsection{Output Description}\label{output-description-23}

You should emit the translated words: Examples:

\begin{verbatim}
31337 -> eleet
storm -> 570R(V)
\end{verbatim}

\subsection{Challenge Input}\label{challenge-input-24}

\begin{verbatim}
I am elite.
Da pain!
Eye need help!
3Y3 (\)33d j00 t0 g37 d4 d0c70r.
1 n33d m4 p1llz!
\end{verbatim}

\subsection{Challenge Output}\label{challenge-output-21}

\begin{verbatim}
I am elite. -> 1 4m 37173
Da pain! -> D4 P41(\)!
Eye need help! -> 3Y3 (\)33D H31P!
3Y3 (\)33d j00 t0 g37 d4 d0c70r. -> Eye need j00 to get da doctor.
1 n33d m4 p1llz! -> I need ma pillz!
\end{verbatim}

\section{Title}\label{title-31}

Lipogram Detector

\subsection{Difficulty}\label{difficulty-30}

Easy

\subsection{Tags}\label{tags-31}

word games, lipogram

\subsection{Description}\label{description-31}

A lipogram is a kind of constrained writing or word game consisting in
writing paragraphs or longer works in which a particular letter or group
of letters is avoided. Writing a lipogram may be a trivial task when
avoiding uncommon letters like Z, J, Q, or X, but it is much more
difficult to avoid common letters like E, T or A, as the author must
omit many ordinary words. A famous example is Poe's poem \emph{The
Raven} contains no Z, but there is no evidence that this was
intentional. Pangrammatic lipograms use all letters except one.

Your challenge today is to detect what letter is missing from the given
text.

\subsection{Input Description}\label{input-description-23}

You'll be given a short piece of text. For example:

\begin{verbatim}
A jovial swain should not complain
Of any buxom fair
Who mocks his pain and thinks it gain
To quiz his awkward air.
\end{verbatim}

\subsection{Output Description}\label{output-description-24}

Your program should emit what letter is missing. From ths above example:

\begin{verbatim}
E
\end{verbatim}

\subsection{Challenge Input}\label{challenge-input-25}

\begin{verbatim}
Bold Nassan quits his caravan,
A hazy mountain grot to scan;
Climbs jaggy rocks to find his way,
Doth tax his sight, but far doth stray.

Not work of man, nor sport of child
Finds Nassan on this mazy wild;
Lax grow his joints, limbs toil in vain-
Poor wight! why didst thou quit that plain?

Vainly for succour Nassan calls;
Know, Zillah, that thy Nassan falls;
But prowling wolf and fox may joy
To quarry on thy Arab boy.
\end{verbatim}

\subsection{Challenge Output}\label{challenge-output-22}

\begin{verbatim}
E
\end{verbatim}

\subsection{Scala Solution}\label{scala-solution-18}

\begin{verbatim}
def lipogram(text: String) : Set[Char] = 
    "ABCDEFGHIJKLMNOPQRSTUVWXYZ".toSet--text.toCharArray.map(_.toUpper).toSet
\end{verbatim}

\subsection{Python Solution}\label{python-solution-2}

\begin{verbatim}
def lipogram(text): 
    return set(string.lowercase) - ( { ch.lower() for ch in text } - set(string.punctuation))
\end{verbatim}

\subsection{Go Solution}\label{go-solution-2}

\begin{verbatim}
package main

import (
    "fmt"
    "gopkg.in/fatih/set.v0"
    "os"
    "strings"
)

func main() {
    const alphabet = "abcdefghijklmnopqrstuvwxyz"
    text := os.Args[1]

    characters := set.New()
    for _, ch := range strings.ToLower(text) {
        characters.Add(string(ch))
    }

    alpha := set.New()
    for _, ch := range alphabet {
        alpha.Add(string(ch))
    }

    fmt.Println(set.Difference(alpha, characters))
}
\end{verbatim}

\section{Title}\label{title-32}

In what year were most presidents alive?

\subsection{Difficulty}\label{difficulty-31}

Easy

\subsection{Tags}\label{tags-32}

dates, presidents

\subsection{Description}\label{description-32}

US presidents serve four year terms, with most presidents serving one or
two terms. Unless a president dies in office, they then live after
leaving office.

This challenge, then, given a list of presidents and their dates of
birth and dates of death, is to figure out what \emph{year} the most
presidents - future, present, or previous - were alive.

\subsection{Challenge Input}\label{challenge-input-26}

Below is a CSV input of presidential birthdates and death dates. Find
what year in which the most number of people who would serve, are
serving, or have served as presidents. The answer might be multiple
years, or only a partial year.

\begin{verbatim}
PRESIDENT,  BIRTH DATE, BIRTH PLACE,    DEATH DATE, LOCATION OF DEATH
George Washington,  Feb 22 1732,    Westmoreland Co. Va.,   Dec 14 1799,    Mount Vernon Va.
John Adams, Oct 30 1735,    Quincy Mass.,   July 4 1826,    Quincy Mass.
Thomas Jefferson,   Apr 13 1743,    Albemarle Co. Va.,  July 4 1826,    Albemarle Co. Va.
James Madison,  Mar 16 1751,    Port Conway Va.,    June 28 1836,   Orange Co. Va.
James Monroe,   Apr 28 1758,    Westmoreland Co. Va.,   July 4 1831,    New York New York
John Quincy Adams,  July 11 1767,   Quincy Mass.,   Feb 23 1848,    Washington D.C.
Andrew Jackson, Mar 15 1767,    Lancaster Co. S.C., June 8 1845,    Nashville Tennessee
Martin Van Buren,   Dec 5 1782, Kinderhook New York,    July 24 1862,   Kinderhook New York
William Henry Harrison, Feb 9 1773, Charles City Co. Va.,   Apr 4 1841, Washington D.C.
John Tyler, Mar 29 1790,    Charles City Co. Va.,   Jan 18 1862,    Richmond Va.
James K. Polk,  Nov 2 1795, Mecklenburg Co. N.C.,   June 15 1849,   Nashville Tennessee
Zachary Taylor, Nov 24 1784,    Orange County Va.,  July 9 1850,    Washington D.C
Millard Fillmore,   Jan 7 1800, Cayuga Co. New York,    Mar 8 1874, Buffalo New York
Franklin Pierce,    Nov 23 1804,    Hillsborough N.H.,  Oct 8 1869, Concord New Hamp.
James Buchanan, Apr 23 1791,    Cove Gap Pa.,   June 1 1868,    Lancaster Pa.
Abraham Lincoln,    Feb 12 1809,    LaRue Co. Kentucky, Apr 15 1865,    Washington D.C.
Andrew Johnson, Dec 29 1808,    Raleigh North Carolina, July 31 1875,   Elizabethton Tenn.
Ulysses S. Grant,   Apr 27 1822,    Point Pleasant Ohio,    July 23 1885,   Wilton New York
Rutherford B. Hayes,    Oct 4 1822, Delaware Ohio,  Jan 17 1893,    Fremont Ohio
James A. Garfield,  Nov 19 1831,    Cuyahoga Co. Ohio,  Sep 19 1881,    Elberon New Jersey
Chester Arthur, Oct 5 1829, Fairfield Vermont,  Nov 18 1886,    New York New York
Grover Cleveland,   Mar 18 1837,    Caldwell New Jersey,    June 24 1908,   Princeton New Jersey
Benjamin Harrison,  Aug 20 1833,    North Bend Ohio,    Mar 13 1901,    Indianapolis Indiana
William McKinley,   Jan 29 1843,    Niles Ohio, Sep 14 1901,    Buffalo New York
Theodore Roosevelt, Oct 27 1858,    New York New York,  Jan 6 1919, Oyster Bay New York
William Howard Taft,    Sep 15 1857,    Cincinnati Ohio,    Mar 8 1930, Washington D.C.
Woodrow Wilson, Dec 28 1856,    Staunton Virginia,  Feb 3 1924, Washington D.C.
Warren G. Harding,  Nov 2 1865, Morrow County Ohio, Aug 2 1923, San Francisco Cal.
Calvin Coolidge,    July 4 1872,    Plymouth Vermont,   Jan 5 1933, Northampton Mass.
Herbert Hoover, Aug 10 1874,    West Branch Iowa,   Oct 20 1964,    New York New York
Franklin Roosevelt, Jan 30 1882,    Hyde Park New York, Apr 12 1945,    Warm Springs Georgia
Harry S. Truman,    May 8 1884, Lamar Missouri, Dec 26 1972,    Kansas City Missouri
Dwight Eisenhower,  Oct 14 1890,    Denison Texas,  Mar 28 1969,    Washington D.C.
John F. Kennedy,    May 29 1917,    Brookline Mass.,    Nov 22 1963,    Dallas Texas
Lyndon B. Johnson,  Aug 27 1908,    Gillespie Co. Texas,    Jan 22 1973,    Gillespie Co. Texas
Richard Nixon,  Jan 9 1913, Yorba Linda Cal.,   Apr 22 1994,    New York New York
Gerald Ford,    July 14 1913,   Omaha Nebraska, Dec 26 2006,    Rancho Mirage Cal.
Jimmy Carter,   Oct 1 1924, Plains Georgia, ,   
Ronald Reagan,  Feb 6 1911, Tampico Illinois,   June 5 2004,    Los Angeles Cal.
George Bush,    June 12 1924,   Milton Mass.,   ,   
Bill Clinton,   Aug 19 1946,    Hope Arkansas,  ,   
George W. Bush, July 6 1946,    New Haven Conn.,    ,   
Barack Obama,   Aug 4 1961, Honolulu Hawaii,    ,
\end{verbatim}

via \href{http://www.presidentsusa.net/birth.html}{U.S. Presidents Birth
and Death Information}.

\section{Title}\label{title-33}

Pandigital Roman Numbers

\subsection{Difficulty}\label{difficulty-32}

Easy

\subsection{Tags}\label{tags-33}

numbers, Roman numerals

\subsection{Description}\label{description-33}

1474 is a pandigital in Roman numerals (MCDLXXIV). It uses each of the
symbols I, V, X, L, C, and M at least once. Your challenge today is to
find the small handful of pandigital Roman numbers up to 2000.

\subsection{Output Description}\label{output-description-25}

A list of numbers. Example:

\begin{verbatim}
1 (I), 2 (II), 3 (III), 8 (VIII) (Examples only, these are not pandigital Roman numbers)
\end{verbatim}

\subsection{Challenge Input}\label{challenge-input-27}

Find all numbers that are pandigital in Roman numerals using each of the
symbols I, V, X, L, C, D and M \emph{exactly} once.

\subsection{Challenge Input Solution}\label{challenge-input-solution-2}

1444, 1446, 1464, 1466, 1644, 1646, 1664, 1666

See \href{http://oeis.org/A105416}{OEIS sequence A105416} for more
information.

\section{Title}\label{title-34}

Pell Numbers

\subsection{Difficulty}\label{difficulty-33}

Easy

\subsection{Tags}\label{tags-34}

number sequence, infinite sequence, number theory

\subsection{Description}\label{description-34}

In mathematics, the
\href{https://en.wikipedia.org/wiki/Pell_number}{Pell numbers} are an
infinite sequence of integers, known since ancient times, that comprise
the denominators of the closest rational approximations to the square
root of 2. This sequence of approximations begins 1/1, 3/2, 7/5, 17/12,
and 41/29, so the sequence of Pell numbers begins with 0, 1, 2, 5, 12,
and 29 (each Pell number is the sum of twice the previous Pell number
and the Pell number before that).

Your challenge today is to generate this sequence and pick out specific
elements in the seqence.

If you're feeling especially brave, try applying memoization and
recursion in your answer.

\subsection{Sample Input}\label{sample-input-2}

You'll be given number \emph{N}, one per line. This is the position in
the sequence of Pell numbers to yield. Examples:

\begin{verbatim}
3
5
\end{verbatim}

\subsection{Sample Output}\label{sample-output-2}

\begin{verbatim}
2
12
\end{verbatim}

\subsection{Challenge Input}\label{challenge-input-28}

\begin{verbatim}
10
17
\end{verbatim}

\subsection{Challenge Output}\label{challenge-output-23}

\begin{verbatim}
985
470832
\end{verbatim}

\subsection{Bonus}\label{bonus-3}

What is the 100th Pell number? (Answer:
27749033099085295754434173207717704165)

\subsection{F\# Solution}\label{f-solution}

\begin{verbatim}
let pell n =
    let addPell a b = (2I*a + b)
    let rec loop n sofar = 
        printfn "%A" sofar
        match ((List.length sofar) = n) with
        | true  -> List.head sofar
        | false -> loop n ((addPell (List.head sofar) (Seq.nth 1 sofar))::sofar)
    match n with
    | 0 -> 0I
    | 1 -> 1I
    | _ -> loop n [1I; 0I]
\end{verbatim}

\section{Title}\label{title-35}

Perfect Numbers

\subsection{Difficulty}\label{difficulty-34}

Easy

\subsection{Tags}\label{tags-35}

divisors, number theory, math, integer sequence

\subsection{Description}\label{description-35}

In number theory, a
\href{http://en.wikipedia.org/wiki/Perfect_number}{perfect number} is a
positive integer that is equal to the sum of its proper positive
divisors, that is, the sum of its positive divisors excluding the number
itself (also known as its aliquot sum). The first perfect number is 6,
because 1, 2, and 3 are its proper positive divisors, and 1 + 2 + 3 = 6.
Equivalently, the number 6 is equal to half the sum of all its positive
divisors: ( 1 + 2 + 3 + 6 ) / 2 = 6. The next perfect number is 28 = 1 +
2 + 4 + 7 + 14. This is followed by the perfect numbers 496 and 8128.

In this challenge you'll be asked to calculate an arbitrary number of
perfect numbers.

\subsection{Input Description}\label{input-description-24}

You'll be given a single integer, \emph{N}, which tells you how many
perfect numbers to emit. Example:

\begin{verbatim}
4
\end{verbatim}

\subsection{Output Description}\label{output-description-26}

Your program should emit a list of perfect numbers up to that point. For
our example:

\begin{verbatim}
6 28 496 8128
\end{verbatim}

\subsection{Challenge Input}\label{challenge-input-29}

\begin{verbatim}
10
\end{verbatim}

\subsection{Challenge Output}\label{challenge-output-24}

\begin{verbatim}
6 28 496 8128 33550336 8589869056 137438691328 2305843008139952128 2658455991569831744654692615953842176 191561942608236107294793378084303638130997321548169216
\end{verbatim}

\subsection{Scala Solution}\label{scala-solution-19}

a naive solution that doesn't use the Euclid-Euler theorem

\begin{verbatim}
def factors(n:Int): List[Int] = (2 to n/2).filter(x => n%x == 0).toList

def perfect(n:Int): Boolean = n == factors(n).sum + 1

(1 to 10000).filter(perfect(_))
\end{verbatim}

getting closer to using Euclid-Euler

\begin{verbatim}
def isprime(n:Int) : Boolean = {
    def check(i:Int) : Boolean = (i > n/2) || ((n % i != 0) && (check (i+1)))
    check(2)
}

def perfect(n:Int): Long = (scala.math.pow(2, (n-1))*(scala.math.pow(2, n)-1)).toLong

(2 to 50000).filter(isprime(_)).map(perfect(_)).distinct
\end{verbatim}

\section{Title}\label{title-36}

Primes in Grids

\subsection{Difficulty}\label{difficulty-35}

Easy

\subsection{Tags}\label{tags-36}

prime numbers

\subsection{Description}\label{description-36}

This puzzle was first proposed (1989) by Gordon Lee: given a grid of
numbers, how many \emph{distinct} primes can you find embedded in the
matrix, regarding that you can read the lines or part of them, in form
vertical, horizontal or diagonal orientation, in both directions.

Note that you can't change direction once you start moving (e.g.~this
isn't Boggle).

\subsection{Input Description}\label{input-description-25}

You'll be given a single number on a line which tells you how many rows
and columns to read (all grids will be square). Example:

\begin{verbatim}
3 
113
754
937
\end{verbatim}

\subsection{Output Description}\label{output-description-27}

Your program should emit the number of distinct primes it finds in the
grid. Optionally list them. Example:

\begin{verbatim}
30
113, 311, 179, 971, 157, 751 359, ...
\end{verbatim}

\subsection{Challenge Input}\label{challenge-input-30}

\begin{verbatim}
5 
11933
99563
89417
33731
32939

6
317333
995639
118142
136373
349199
379379
\end{verbatim}

\subsection{Challenge Output}\label{challenge-output-25}

\begin{verbatim}
116

187
\end{verbatim}

\section{Title}\label{title-37}

Reverse Factorial

\subsection{Difficulty}\label{difficulty-36}

Easy

\subsection{Tags}\label{tags-37}

math, factorial

\subsection{Description}\label{description-37}

Nearly everyone is familiar with the factorial operator in math. 5!
yields 120 because factorial means ``multiply successive terms where
each are one less than the previous'':

\begin{verbatim}
5! -> 5 * 4 * 3 * 2 * 1 -> 120
\end{verbatim}

Simple enough.

Now let's reverse it. Could you write a function that tells us that
``120'' is ``5!''?

Hint: The strategy is pretty straightforward, just divide the term by
successively larger terms until you get to ``1'' as the resultant:

\begin{verbatim}
120 -> 120/2 -> 60/3 -> 20/4 -> 5/5 -> 1 => 5!
\end{verbatim}

\subsection{Sample Input}\label{sample-input-3}

You'll be given a single integer, one per line. Examples:

\begin{verbatim}
120
150
\end{verbatim}

\subsection{Sample Output}\label{sample-output-3}

Your program should report what each number is as a factorial, or
``NONE'' if it's not legitimately a factorial. Examples:

\begin{verbatim}
120 = 5!
150   NONE
\end{verbatim}

\subsection{Challenge Input}\label{challenge-input-31}

\begin{verbatim}
3628800
479001600
6
18
\end{verbatim}

\subsection{Challenge Output}\label{challenge-output-26}

\begin{verbatim}
3628800 = 10!
479001600 = 12!
6 = 3!
18  NONE
\end{verbatim}

\subsection{Fsharp Solution}\label{fsharp-solution-2}

\begin{verbatim}
let rec tcaf(n: int) (sofar: int) =
    match (n%sofar) with 
    | 0 ->  match (n/sofar) with
            | 1 -> sofar
            | _ -> tcaf (n/sofar) (sofar+1)
    | _ -> -1

let solution (n: int) = 
    let res = tcaf n 2
    match res with
    | -1 -> "NONE"
    | _  -> (string res) + "!"
\end{verbatim}

\section{Title}\label{title-38}

Roller Coaster Words

\subsection{Difficulty}\label{difficulty-37}

Easy

\subsection{Tags}\label{tags-38}

word play

\subsection{Description}\label{description-38}

A
\href{http://www.questrel.com/records.html\#spelling_alphabetical_order_entire_word_roller-coaster}{roller
coaster word} is a word with letters that alternate between going
forward and backward in alphabet. One such word is
``decriminalization''. Can you find other examples of roller coaster
words in the English dictionary?

\subsection{Output}\label{output-1}

Your program should emit any and all roller coaster words it finds in a
standard English language dictionary longer (or
\href{https://github.com/dolph/dictionary/blob/master/enable1.txt}{enable1.txt})
than 4 letters.

\subsection{Notes}\label{notes-1}

If you have your own idea for a challenge, submit it to
/r/DailyProgrammer\_Ideas, and there's a good chance we'll post it.

\section{Title}\label{title-39}

Safe Prime Numbers

\subsection{Difficulty}\label{difficulty-38}

Easy

\subsection{Tags}\label{tags-39}

prime numbers, number theory, math, encryption

\subsection{Description}\label{description-39}

A \href{https://en.wikipedia.org/wiki/Safe_prime}{safe prime} is a prime
number of the form 2\emph{p} + 1, where p is also a prime. Safe primes
are also important in cryptography because of their use in discrete
logarithm-based techniques like Diffie-Hellman key exchange.

\subsection{Input Description}\label{input-description-26}

You will be given a single number that is the maximum value of safe
prime to search for. Example:

\begin{verbatim}
100
\end{verbatim}

\subsection{Output Description}\label{output-description-28}

A list of numbers, one on each line, showing numbers that solve the safe
prime definition. Example:

\begin{verbatim}
5
7
11
23
47
59
83
\end{verbatim}

\subsection{Challenge Input}\label{challenge-input-32}

\begin{verbatim}
1000
\end{verbatim}

\subsection{Challenge Input Solution (not visible by
default)}\label{challenge-input-solution-not-visible-by-default}

\begin{verbatim}
5
7
11
23
47
59
83
107
167
179
227
263
347
359
383
467
479
503
563
587
719
839
863
887
983
\end{verbatim}

\subsection{FSharp Solution}\label{fsharp-solution-3}

\begin{verbatim}
> let isprime (n:int) =                                                             
-     let n = bigint(n)
-     let rec check i =
-         i > n/2I || (n % i <> 0I && check (i + 1I))
-     check 2I
-- ;;


> let safeprimes(n:int) =
-     [ 2..n ] |> List.filter (fun x ->isprime(x) && isprime(1+2*x) )
- ;;

val safeprimes : n:int -> int list

> safeprimes 100 ;;
val it : int list = [2; 3; 5; 11; 23; 29; 41; 53; 83; 89]
> safeprimes 1000 ;;
val it : int list =
  [2; 3; 5; 11; 23; 29; 41; 53; 83; 89; 113; 131; 173; 179; 191; 233; 239; 251;
   281; 293; 359; 419; 431; 443; 491; 509; 593; 641; 653; 659; 683; 719; 743;
   761; 809; 911; 953]
\end{verbatim}

\section{Title}\label{title-40}

Typo Maker

\subsection{Difficulty}\label{difficulty-39}

Easy

\subsection{Tags}\label{tags-40}

word play

\subsection{Description}\label{description-40}

Typos are great fun, and often follow a common pattern - keys next to
eachother, doubled or omitted characters, and transpositions. Can you
write a program to generate common typos? If so, you could be on your
way to typo-squatting in DNS!

Common typos fall into a few different categories:

\begin{itemize}
\itemsep1pt\parskip0pt\parsep0pt
\item
  Skip letter
\item
  Double letters
\item
  Reverse (transpose) letters
\item
  Skip spaces
\item
  Missed key
\item
  Inserted key
\end{itemize}

For this challege, when you think about neighbor keys, assume a
{[}QWERTY keyboard layout{]} (http://en.wikipedia.org/wiki/QWERTY).

\subsection{Input Description}\label{input-description-27}

You'll be given a word, one per line, and asked to generate typos for
it. Example:

\begin{verbatim}
typo
\end{verbatim}

\subsection{Output Description}\label{output-description-29}

Your program should emit the word mangled into its various formats using
the above categories. Our example:

\begin{verbatim}
tpo
ypo
typ
ttypo
tyypo
typpo
typoo
tyop
ytpo
toyp
rypo
yypo
ttpo
tupo
ty[o
tyoo
typi
typp
trypo
rtypo
... (omitted for brevity)
\end{verbatim}

\subsection{Input Description}\label{input-description-28}

\begin{verbatim}
Facebook
Google
Global thermonuclear war
Dead as a doornail
Britney Spears
Cappuccino
Everybody to the limit
\end{verbatim}

\section{Title}\label{title-41}

Wedderburn-Etherington Sequence

\subsection{Difficulty}\label{difficulty-40}

Easy

\subsection{Tags}\label{tags-41}

integer sequence, infinite sequence, math, number theory

\subsection{Description}\label{description-41}

The Wedderburn-Etherington numbers are an integer sequence named for
Ivor Malcolm Haddon Etherington and Joseph Wedderburn that can be used
to count certain kinds of binary trees. The first few numbers in the
sequence are

\begin{verbatim}
0, 1, 1, 1, 2, 3, 6, 11, 23, 46, 98, 207, 451 ...
\end{verbatim}

The Wedderburnâ--Etherington numbers may be calculated using the
recurrence relation (in LaTeX notation)

\begin{verbatim}
a_{2n-1} = \sum\limits{i=1}^n-1 a_ia_{2n-i-1}

a_{2n} = a_n(a_n+1)/2 + \sum\limits{i=1}^n-1 a_ia_{2n-i}
\end{verbatim}

See Wikipedia for more on the
\href{http://en.wikipedia.org/wiki/Wedderburn\%E2\%80\%93Etherington_number}{Wedderburn-Etherington
number} and its uses.

\subsection{Input Description}\label{input-description-29}

You'll be given a number \emph{n}, the number in the sequence to
generate.

\subsection{Output Description}\label{output-description-30}

A sequence of integers in the Wedderburn-Etherington sequence up to
position \emph{n}.

\section{Title}\label{title-42}

XOR Multiplication

\subsection{Difficulty}\label{difficulty-41}

Easy

\subsection{Tags}\label{tags-42}

math, multiplication, XOR

\subsection{Description}\label{description-42}

One way to think about bitwise \emph{addition} (using the symbol
\texttt{\^{}}) as binary addition without carrying the extra bits:

\begin{verbatim}
   101   5
^ 1001   9
  ----  
  1100  12

  5^9=12
\end{verbatim}

So let's define XOR multiplcation (we'll use the symbol \texttt{@}) in
the same way, the addition step doesn't carry:

\begin{verbatim}
     1110  14
   @ 1101  13
    -----
     1110
       0
   1110
^ 1110 
  ------
  1000110  70

  14@13=70
\end{verbatim}

For this challenge you'll get two non-negative integers as input and
output or print their XOR-product, using both binary and decimal
notation.

\subsection{Input Description}\label{input-description-30}

You'll be given two integers per line. Example:

\begin{verbatim}
5 9
\end{verbatim}

\subsection{Output Description}\label{output-description-31}

You should emit the equation showing the XOR multiplcation result:

\begin{verbatim}
5@9=12
\end{verbatim}

\subsection{Challenge Input}\label{challenge-input-33}

\begin{verbatim}
1 2
9 0
6 1
3 3
2 5
7 9
13 11
5 17
14 13
19 1
63 63
\end{verbatim}

\subsection{Challenge Output}\label{challenge-output-27}

\begin{verbatim}
1@2=2
9@0=0
6@1=6
3@3=5
2@5=10
7@9=63
13@11=127
5@17=85
14@13=70
19@1=19
63@63=1365
\end{verbatim}
